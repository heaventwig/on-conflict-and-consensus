% Spelling checked 2004-12-14
Originally, C.T. wrote this book for the Pledge of Resistance in Boston when it had over 3500 signers and 150 affinity groups. All policy decisions for the organization were made at monthly spokes\index{meeting}meetings, involving at least one spokesperson from each affinity group. Members from the coordinating committee were charged with managing daily affairs. Spokes\index{meeting}meetings were often attended by over one hundred people; they were usually seventy strong. For almost two years the process of consensus worked well for the Pledge, empowering very large numbers of people to engage confidently in nonviolent direct action. The forerunner of the model of consensus outlined in this book was used throughout this period at spokes\index{meeting}meetings and, particularly well, at the weekly coordinators \index{meeting}meetings. However, it was never systematically defined and written down or formally adopted. 

For over two years, C.T. attended monthly spokesmeetings, weekly coordinating \index{meeting}meetings, and uncounted committee \index{meeting}meetings. He saw the need to develop a consistent way to introduce new members to consensus. At first, he looked for existing literature to aid in conducting workshops on the consensus process. He was unable to find any suitable material, so he set out to develop his own.

The first edition of this book is the result of a year of research into consensus in general and the Pledge process in particular. It was mostly distributed to individuals who belonged to various groups already struggling to use some form of consensus process. The fourth printing included an \index{introduction}introduction which added the concept of secular consensus. The secular label distinguishes this model of consensus from both the more traditional model found in faith-based communities and the rather informal consensus commonly found in progressive groups.

Unfortunately, the label of secular consensus gave the impression that we were denying any connection with spirituality. We wanted to clearly indicate that the model of consensus we were proposing was distinct, but we did not want to exclude the valuable work of faith-based communities.

Therefore, since the sixth printing we have used the name Formal Consensus because it adequately defines this distinction. We hope that Formal Consensus will continue to be an important contribution to the search for an effective, more unifying, democratic \index{decisionmaking}decisionmaking process.

Formal Consensus is a specific kind of \index{decisionmaking}decisionmaking. It must be defined by the group using it. It provides a foundation, \index{structure}structure, and collection of \index{techniques}techniques for efficient and productive group discussions. The foundation is the commonly-held principles and decisions which created the group originally. The \index{structure}structure is predetermined, although flexible. The \index{agenda}agenda is formal and extremely important. the \index{role}roles, \index{techniques}techniques, and skills necessary for smooth operation must be accessible to and developed in all members. \index{evaluation}Evaluation of the process must happen on a consistent and frequent basis, as a tool for self-education and self-management. Above all, Formal Consensus must be taught. It is unreasonable to expect people to be familiar with this process already. In general, nonviolent conflict resolution does not exist in modern North American society. These skills must be developed in what is primarily a competitive environment. Only time will tell if, in fact, this model will flourish and prove itself effective and worthwhile. 

We are now convinced more than ever that the model presented in this book is profoundly significant for the future of our species. We must learn to live together cooperatively, resolving our conflicts nonviolently and making our decisions consensually. We must learn to value diversity and \index{respect}respect all life, not just on a physical level, but emotionally, intellectually, and spiritually. We are all in this together.
\begin{flushright}C.T. Butler\\
Amy Rothstein\\
August 1991
\end{flushright}