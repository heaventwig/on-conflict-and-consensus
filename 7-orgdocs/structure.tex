%Spelling checked 2004-04-02
\section*{Preface}

The following \index{structure}structure proposal is designed for an organization of 100 people or more.

Each member of the organization would belong to an affinity group of between 5 and 20 adults. Each affinity group would be autonomous, having complete authority to decide who is in the affinity group, how the affinity group is organized internally, and to what degree the affinity group directly participates in the organization.

Affinity groups can be created by a group of new members, a split or spin off from an existing affinity group, or individuals drawn from other affinity groups into a new affinity group. The relationships between members of an affinity group are created by the people involved. In this organization, behaviors are expected to be nonviolent, \index{respect}respectful, and egalitarian.

This organization of affinity groups accepts and promotes diverse and possibly challenging types of affinity groups. No affinity group will be denied \index{participation}participation in the organization because of sexual orientation, ethnic or cultural background, age, physical or mental difference.

This \index{structure}structure outline is based upon two other documents, the Community X Vision Statement and the Community X Principles and Values.

If you have questions, please call for Formal Consensus Technical Assistance at 1-800-569-4054.

\section*{Organizational \index{structure}structure outline}

\begin{enumerate}
\item \index{meeting}Meetings

\begin{enumerate}
\item Affinity \index{meeting}Meetings
\item Annual Community \index{meeting}Meetings
\item Elders Council
\item Peace Council
\item Orientation \index{meeting}Meetings
\item Trainings
\item Committee \index{meeting}Meetings
\item Special \index{meeting}Meetings
\end{enumerate}

\item Committees

\begin{enumerate}
\item Executive
\item Financial
\item Membership
\item Newsletter
\item Outreach
\item Fundraising
\item Community Building
\item \index{agenda}Agenda Planning
\item Directory
\item Childcare
\item Healthcare
\item Affinity Groups
\item Mediation
\item Festival
\end{enumerate}

\item \index{role}Roles

\begin{enumerate}
\item \index{meeting}Meeting Specific
\begin{enumerate}
\item \index{facilitator}Facilitator
\item Notetaker
\item Public Scribe
\item Timekeeper
\item Greeter
\item Peacekeeper
\item Refreshment Coordinator/Housekeeper
\end{enumerate}
\item Long Term
\begin{enumerate}
\item Treasurer
\item \index{agenda}Agenda \& Schedule Coordinator
\item Archivist
\item Newsletter Editor
\end{enumerate}
\end{enumerate}
\end{enumerate}

\section*{\index{meeting}Meetings}
All \index{meeting}meetings of Community X are open to all community members.  All \index{meeting}meetings are conducted in the spirit of nonviolent conflict resolution, with special attention given to open-mindedness so that diversity of ideas and lifestyles is valued and supported.  This is accomplished by using Formal Consensus \index{decisionmaking}decisionmaking which provides for appropriate conflict and the ability to ``agree to disagree''.  All decisions at any \index{meeting}meeting must be in harmony with the Principles and Values and Vision Statement of Community X and consistent with any previous decisions.

\subsection*{Affinity \index{meeting}Meetings}
All business, financial, and policy implementation decisions are made at monthly affinity group business \index{meeting}meetings. (This business \index{meeting}meeting is in addition to other kinds of affinity group get-togethers each month and may be a part of a longer affinity group event.)  All members of the affinity group are expected to attend. Non-members are welcome to observe or give reports when invited. A quorum is attained when at least two thirds of the current affinity group members are present.  It is suggested that decisions be made by consensus using Formal Consensus but each affinity group can decide what process they want to use internally. Notes are taken at every affinity group \index{meeting}meeting. The \index{notetaker}notetaker is expected to reproduce and distribute the notes in a timely fashion, but no later than 10 days before the next \index{meeting}meeting. In addition, a copy of the notes for each affinity group \index{meeting}meeting will be kept on permanent file with the community archivist. Access to these notes will be granted to any Community X member upon request.

\subsection*{Annual Community \index{meeting}Meeting}
The Community \index{meeting}meeting occurs at least once each year, generally in the summer. All members of the community are invited to attend. Notice of this \index{meeting}meeting will be mailed to all members at least one month in advance of the \index{meeting}meeting. A quorum is at least 10\% of the current members. A \index{meeting}meeting will be deemed adjourned if a quorum is lost. All decisions are made by consensus using Formal Consensus (as defined in \emph{On Conflict and Consensus}, including adaptations defined in \emph{Formal Consensus at Community X \index{meeting}Meetings}). The community \index{meeting}meeting is the highest decision making body in Community X. All major policy decisions which effect every affinity group in Community X need to be passed by this \index{meeting}meeting. Any decision made by another part of Community X can be reviewed and/or overturned by the members at a community \index{meeting}meeting. Only members can raise concerns. A minimum of three volunteer members will be appointed to each committee at the annual community \index{meeting}meeting. Notes will be taken at every community \index{meeting}meeting and kept on permanent file with the archivist. Access to these notes will be granted to any Community X community member upon request. The cost for copying and mailing is the responsibility of the member.

\subsection*{Elders Council}
Each affinity group will appoint one member to the Elders Council. (The suggested qualities are long term involvement with Community X and a wise, calm, fair-minded personality.) Elders Council will meet as necessary, but at least once a year to elect a convener and review the state of the community. All decisions are made by consensus using Formal Consensus. Any elder can request the Elders Council convene. The convener will organize an Elders Council within one month of such a request. A quorum will be attained when there is an Elder from at least 75\% of the affinity groups in the community. The Elders Council is responsible for resolving conflicts between affinity groups, between an affinity group and another organization (when possible), and interpret Community X policy decisions. Decisions by the Elders Council are final except they can be reviewed and overturned by a consensus of the members at a community \index{meeting}meeting.


\subsection*{Peace Council}
Any member of the Community X may volunteer for the Peace Council. The Peace Council will meet as necessary, but at least once a year to elect a convener and develop plans for nonviolence and mediation trainings for the community. At any time, any member may request a mediation or ``Peace Council'' to resolve a conflict she or he has been unable to resolve directly with another member. The member may contact someone who has identified him or herself as a mediator (or member of the Peace Council) and ask the mediator to arrange a mediation with the person in conflict. If the member cannot find or does not know a mediator, she or he may contact the convener of the Peace Council, who will assist them in finding a suitable mediator. In all situations, \index{participation}participation in a mediation by all parties is voluntary, including the mediator. Generally, out of \index{respect}respect for the participants involved, a mediation is a private matter and should remain that way. If the conflict is unable to be resolved through mediation, the people involved may appeal to the Elders Council or the community \index{meeting}meeting.


\subsection*{Orientation \index{meeting}Meetings}
The Orientation \index{meeting}Meeting takes place as often as needed for the orientation of new members into Community X. It will include (but is not limited to): a history of Community X; the decision making \index{structure}structure of Community X; an \index{introduction}introduction to the Formal Consensus \index{decisionmaking}decisionmaking process of Community X; a brief lesson on nonviolence and community; a brief exploration into oppression and diversity; and some fun and games. The membership committee will be responsible for organizing these \index{meeting}meetings. These orientations will be open to all Community X members.


\subsection*{Trainings}
There are at least three areas in which all of us need regular training: consensus \index{decisionmaking}decisionmaking, nonviolence, and ending oppression. Community X will offer trainings in each of these areas at least once a year. The Membership Committee will be responsible for organizing these trainings. All Community X members are encouraged to attend these trainings.


\subsection*{Committee \index{meeting}Meetings}
There are fourteen standing committees. (See the section on committees for more details on each of these committees.). In addition, there will be ad hoc committees, as needed, created by the members, the affinity groups, or at community \index{meeting}meetings. All members of Community X are expected to be on at least one committee. A quorum at each committee \index{meeting}meeting is at least three Community X members. Each committee will meet as often as necessary to conduct its business. Each committee will be responsible for carrying out the tasks assigned to it and will report back to the sponsoring body on its work. Any decisions proposed by any committee must be brought to an affinity group or community \index{meeting}meeting and put on the \index{agenda}agenda. No committee has authority to make decisions in the name of Community X unless explicitly charged with that responsibility for a particular decision by the Executive Committee or the community \index{meeting}meeting.


\subsection*{Special \index{meeting}Meetings}
From time to time, the Executive committee might decide to hold a Special Community \index{meeting}Meeting. This \index{meeting}meeting may be empowered to make decisions for Community X if all current members are notified of the \index{meeting}meeting \index{agenda}agenda, time, and place, by mail, at least 14 days prior to the \index{meeting}meeting. Any affinity group, council, or committee may hold special \index{meeting}meetings as desired so long as all current members of that body are notified in a timely fashion.


\section*{Committees}
There are fourteen standing committees. Ad hoc committees can be convened as needed by the members, the affinity groups, or at community \index{meeting}meetings. Membership on any committee is open to all members of the Community X community. A minimum of three volunteer members will be appointed to each committee at the annual community \index{meeting}meeting. Internal \index{structure}structure and process for each committee is determined by each committee. Each committee will be responsible for submitting an annual operating budget (January 1 - December 31) to the Treasurer by the end of September each year.


\subsection*{Executive}
The Executive Committee is composed of the Treasurer, the \index{agenda}Agenda \& Schedule Coordinator, the current convener of the Elders Council, and two additional past or present Elders appointed at the annual community \index{meeting}meeting. There are no regularly scheduled \index{meeting}meetings of the Executive Committee. It meets only as needed and can be convened by any one of its members. The Executive Committee is responsible for oversight and coordination of the annual community \index{meeting}meeting, including appointing the convener of it. It is charged with fulfilling the decisions made at the community \index{meeting}meetings. The executive committee cannot make policy; however, it can, when necessary, make decisions in the name of Community X for which there is no existing policy decision or about which the policy decision is unclear. Any decision made in this manner is provisional until the next scheduled community \index{meeting}meeting, when the decision will be revisited and a consensus must be reached for the decision to stand.

\subsection*{Financial}
The Financial Committee is responsible for general fiscal management, including oversight of the Treasurer and the annual budget. The Financial Committee recommends an annual budget to the community \index{meeting}meeting for approval in October for the following calendar year. The financial committee meets as often as needed but at least twice each year; once in early October to create a proposed annual budget, and once immediately following the annual community \index{meeting}meeting (where at least three members volunteer for this committee for the coming year).


\subsection*{Membership}
The Membership Committee is a minimum of three volunteer members appointed at the annual community \index{meeting}meeting. This committee will organize and facilitate Orientation \index{meeting}Meetings. The committee will keep records of attendance at Orientation \index{meeting}Meetings. The Membership Committee implements the membership policies of Community X. As a minimum, requirements for becoming a member are: 1) attend an Orientation \index{meeting}Meeting; 2) attend at least one Community X sponsored event each year; 3) volunteer for at least five hours of service to the Community X community each year (not including volunteer work at Community X events); and, 4) regularly attend the \index{meeting}meetings of at least one committee. [Note: Community X makes available financial assistance for all Community X events so that no one is unable to become a member because of a financial barrier.] The Membership Committee will consider exemptions for those who have challenges and are unable to fulfill these requirements. It will be the Membership Committee's responsibility to record and maintain up-to-date information on member's volunteer time and evaluate fulfillment of the membership requirement for each member on an annual basis.


\subsection*{Newsletter}
The Newsletter Committee is responsible for production and distribution of the newsletter on a regular basis. The editorial policy will be determined by this committee. [Note: As a minimum editorial policy, no submission will be accepted for printing which advocates or encourages the use of violence. Also, an article may be edited so it will fit available space. No changes will be made to a submission without the author's permission.]


\subsection*{Outreach}
The Outreach Committee is responsible for educating the general public about how to become a member of Community X and encouraging people to join. The Outreach Committee is responsible for promoting Community X events. (However, since this must be done with some sensitivity to the nature of Community X and because no policy or precedent currently exists, before this committee can begin promoting Community X events, they must develop and propose guidelines for community approval.)


\subsection*{Fundraising}
The task of the Fundraising Committee is the planning and implementation of activities which raise money for the operating expenses of Community X. This may include: dues, donations collected at events, direct appeals through the newsletter or the mail, special events, performances, the annual festival, Community X merchandise sales, and grants. The challenge for this committee is creating a fundraising component at every Community X event.


\subsection*{Community Building}
The Community Building Committee is responsible for developing community within Community X. This might include: conducting rituals; organizing group excursions; creating a community center; establishing a community land trust; managing a community educational resource library; etc.


\subsection*{\index{agenda}Agenda Planning}
The \index{agenda}Agenda Planning Committee is composed of the \index{agenda}Agenda \& Schedule Coordinator, the \index{facilitator}facilitator and convener of community \index{meeting}meetings, and any other member. Together they create a proposed \index{agenda}agenda for community \index{meeting}meetings. In addition, they assist the convener in organizing community \index{meeting}meetings.


\subsection*{Directory}
The Directory Committee compiles and publishes the Community X directory on an annual basis.


\subsection*{Childcare}
The Childcare committee organizes childcare for every Community X event and is a resource for affinity groups needing childcare for their activities.


\subsection*{Healthcare}
The Healthcare committee provides resources and support for quality alternative healthcare opportunities for members of Community X.


\subsection*{Affinity Groups}
The Affinity Groups committee assists in organizing new affinity groups and matching new members with appropriate affinity groups.


\subsection*{Mediation}
The mediation committee collects resources and organizes trainings in conflict resolution skills and keeps an up-to-date profile of all members of Community X who are willing to mediate.


\subsection*{Festival}
The Festival committee is responsible for organizing at least one ``open to the public'' festival each year for Community X, usually in the summer and coinciding with the annual community \index{meeting}meeting.


\section*{\index{role}Roles}
Any of the following \index{role}roles may be filled by a member of Community X.


\subsection*{\index{meeting}Meeting Specific}
These \index{role}roles are specifically for the annual community \index{meeting}meetings. Affinity groups and committee \index{meeting}meetings can choose to use similar \index{role}roles or may define their own. \index{role}Roles are filled by volunteers at the end of each \index{meeting}meeting for the next \index{meeting}meeting and published in the notes. If more than one person volunteers for a \index{role}role and they cannot decide among themselves who will take the \index{role}role, there will be an open vote with the person getting the most votes getting the \index{role}role. Everyone is encouraged to fill any \index{role}role. To facilitate this, it will be a general guideline that a \index{role}role will be offered to someone who has not yet filled it before it is open to others who have filled it before; and, that no one is expected to fill a \index{role}role more than once every year. At any time, a person, especially a new member, may ask for an experienced partner to assist her/him in fulfilling the \index{role}role.


\subsubsection*{\index{facilitator}Facilitator}
(See \emph{On Conflict and Consensus}, Section \ref{role:facilitator}.) The \index{facilitator}facilitator is expected to meet with the \index{agenda}Agenda Planning Committee prior to the community \index{meeting}meeting to plan a proposed \index{agenda}agenda and \index{brainstorming}brainstorm on discussion \index{techniques}techniques for specific \index{agenda}agenda items.

\subsubsection*{\index{notetaker}Notetaker}
(See \emph{On Conflict and Consensus}, Section \ref{role:notetaker}.) In addition, the Note Taker records the attendance at community \index{meeting}meetings and who filled which \index{role}role.

\subsubsection*{\index{public scribe}Public Scribe} % (fold)
%\label{ssub:public_scribe}
(See \emph{On Conflict and Consensus}, Section \ref{role:public_scribe}.)
% subsubsection public scribepublic_scribe (end)

\subsubsection*{\index{timekeeper}Timekeeper}
(See \emph{On Conflict and Consensus}, Section \ref{role:timekeeper}.)

\subsubsection*{\index{doorkeeper}Greeter}
(See \emph{On Conflict and Consensus}, Section \ref{role:greeter}.)

\subsubsection*{\index{peacekeeper}Peacekeeper}
(See \emph{On Conflict and Consensus}, Section \ref{role:peacekeeper}.)

\subsubsection*{Refreshment Coordinator/Housekeeper}
The Refreshment Coordinator provides a meal at each community \index{meeting}meeting. In keeping with tradition, the meal should be vegan (no animal products). The cost will be paid by Community X. The Housekeeper is responsible for the physical environment of the \index{meeting}meeting space.

\subsection*{Long Term}
The long term \index{role}roles are appointed annually at the community \index{meeting}meeting for the following year. Volunteers must be members and willing to accept the \index{role}role for the entire year.

\subsubsection*{Treasurer}
The Treasurer is responsible for managing the Community X money and checking account, keeping the financial records, cutting checks, and assisting in creating the annual budget. The Treasurer is expected to attend community \index{meeting}meetings and provide written quarterly financial reports to the Executive Committee and filed with the Archivist. On occasion, the Treasurer may be delegated to make a decision about some business or financial matter by the Executive Committee when expedient, necessary, or desirable.

\subsubsection*{\index{agenda}Agenda \& Schedule Coordinator (ASC)}
The ASC is responsible for keeping a schedule of all Community X functions. The primary purpose is information sharing to avoid scheduling conflicting events whenever possible. The ASC is expected to attend community \index{meeting}meetings. Also, the ASC is expected to meet with the \index{facilitator}facilitator prior to each community \index{meeting}meeting to create a proposed \index{agenda}agenda and \index{brainstorming}brainstorm on discussion \index{techniques}techniques for specific \index{agenda}agenda items. In addition, the ASC is responsible for keeping track of tabled items which need to be included on the \index{agenda}agenda at the next \index{meeting}meeting. Likewise, the ASC is responsible for tracking the \index{evaluation}evaluation comments and noticing if the same or similar issues keep recurring. When this happens, the ASC might place this issue on the \index{agenda}agenda for discussion or take whatever other appropriate action necessary to address and resolve the issue.

\subsubsection*{Archivist}
The Archivist is responsible for the archives of Community X. The archives should include a copy of all documents generated by the community \index{meeting}meetings, including the notes from each \index{meeting}meeting, and other significant material important to the life of the organization.

\subsubsection*{Newsletter Editor}
The Newsletter Editor is responsible for publishing and distributing the Newsletter for the Community X community.

%\begin{flushleft}Created by: C.T. Butler\end{flushleft}%
%\begin{flushleft}
Food Not Bombs Publishing\\
7304 Carroll Ave \#136\\
Takoma Park, MD 20912\\
1-800-569-4054\\
email: ctbutler@consensus.net
\end{flushleft}