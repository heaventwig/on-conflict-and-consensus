%Spelling checked 2004-04-06
\newtheorem{art}{Article}
\newtheorem{asec}{}[art]
%Generic By Laws


Bylaws of Community X, Inc.


\begin{art}[Name and Location]

Our official name is COMMUNITY X, INCORPORATED, (hereinafter
Community X), and the location of our principal office is Suite
123, 456 Forest Avenue, Anywhere, USA.

\end{art}

\begin{art}[Purpose] ~

\begin{asec}[Charitable, Humane Organization]


Community X is organized exclusively for charitable purposes:
\begin{enumerate}
\item (First charitable purpose)
\item (Second charitable purpose)
\item (Third charitable purpose)
\end{enumerate}
\end{asec}

\begin{asec}[Exclusively Nonprofit, Tax-exempt Activities]

Notwithstanding any other provision of these articles, the
corporation shall not carry on any other activities not permitted
to be carried on (a) by a corporation exempt from federal income
tax under section 501 (c) (3) of the Internal Revenue Code, or
corresponding section of any future federal tax code, or (b) by a
corporation, contributions to which are deductible under section
170 (c) (2) of the Internal Revenue Code, or corresponding section
of any future federal tax code.

\end{asec}
\end{art}

\begin{art}[Fiscal Year]

The fiscal year of the corporation shall run from January 
1 until December 31.

\end{art}


\begin{art}[Board of Directors] ~

\begin{asec}[Number]
 
The Board of Directors shall consist of all members
in good standing.

\end{asec}


\begin{asec}[Tenure, Renewal Terms, and Removal]

A director's tenure begins at the Board \index{meeting}meeting immediately
following the certification by the Membership Committee that this
person has successfully fulfilled the requirements of membership.
Every other annual \index{meeting}meeting, each and every director's tenure is
renewed by consent of those present. In order to terminate a
director's tenure for cause, a proposal to terminate said tenure
must be made by another director at an annual or regular Board
\index{meeting}meeting.  The director whose tenure may be terminated must be
given reasonable notice and an opportunity to be heard at the
\index{meeting}meeting considering her or his termination.  Provided a quorum is
present, a consensus in favor of the director's removal shall
cause said director's tenure to be terminated.  The director whose
tenure may be terminated does not participate in the call for
consensus on the issue of her or his termination.

\end{asec}

\begin{asec}[\index{power}Powers of the Board]

The affairs of the corporation shall be managed by the directors
who shall have and may exercise all the \index{power}powers of the corporation,
including but not limited to: a) approving all proposals and
applications for funding; b) entering into agreements and
contracts consistent with the purposes of the corporation; c)
hiring the staff; d) electing the Chairperson President,
Treasurer, and Clerk of the corporation at the annual directors'
\index{meeting}meeting.

\end{asec}

\begin{asec}[Annual \index{meeting}Meeting and regular Board \index{meeting}Meeting]

There shall be an annual \index{meeting}meeting of directors on the third Sunday
in August, where the officers of the corporation for the upcoming
fiscal year shall be elected and all regular business and policy
making shall occur.  Additional regular Board \index{meeting}meetings or
committee \index{meeting}meetings may be held as needed.

\end{asec}
\begin{asec}[Special Board \index{meeting}Meetings]

Special Board \index{meeting}meetings may be called at any time by consent of ten
percent (10\%) of current directors in good standing.

\end{asec}
\begin{asec}[Notice]

Reasonable notice to all directors must be given
for all \index{meeting}meetings.  Two week's notice via e-mail, letter, or phone,
shall be considered reasonable notice.  In the case of a
``special'' \index{meeting}meeting called in an emergency, forty-eight hours
notice shall be considered reasonable.

\end{asec}
\begin{asec}[Quorum]

At any directors \index{meeting}meeting, the attendance of at least ten percent
(10\%) of the directors in good standing shall constitute a
quorum.

\end{asec}
\begin{asec}[Action by Formal Consensus]

When a quorum is present at any \index{meeting}meeting, a consensus, using Formal
Consensus (as defined in On Conflict and Consensus by
C.T. Lawrence Butler and Amy Rothstein) shall decide any question.

\end{asec}
\begin{asec}[Compensation]

The Board may from time to time determine in good faith, to
compensate directors for their services, which may include
expenses of attendance at \index{meeting}meetings.  Directors shall not be
precluded form serving the corporation in any other capacity and
receiving compensation for any such services.

\end{asec}
\begin{asec}[Committees]

The directors, by consent, may elect or appoint one or more
committees and may delegate to any such committee or committees
any or all of their \index{power}powers.  The committee shall remain operative
as long as it is deemed necessary by the directors.

\end{asec}
\end{art}
\clearpage

\begin{art}[Officers of the Corporation] ~
\begin{asec}[Election]

The president, treasurer, and clerk of the corporation shall be
elected by the directors at the annual \index{meeting}meeting of the directors.
Only members in good standing may be elected officers.  Further,
the clerk shall be a resident of the state of incorporation.  In
addition, the directors may elect a convener, one or more
\index{facilitator}facilitators and such assistant clerks and assistant treasurers as
it may deem proper.  No more than one office may be held by the
same person.
\end{asec}

\begin{asec}[Tenure]

Officers' terms are for one year, and until the succeeding officer
is chosen and qualified.

\end{asec}

\begin{asec}[Renewal Terms]

Any or all of the officers may be elected for renewal terms by the
consent of the directors.

\end{asec}

\begin{asec}[Removal of Officers]

Any officers' tenure may be terminated for cause by consent of the
directors provided reasonable notice is given and the officer has
an opportunity to speak at the directors \index{meeting}meeting where her or his
termination is being considered.

\end{asec}

\begin{asec}[\index{facilitator}Facilitator of the Board]

The \index{facilitator}facilitator shall preside at all directors \index{meeting}meetings, and shall
have and perform such duties as may be assigned to her or him by
the directors.

\end{asec}

\begin{asec}[President of the Corporation]

The president shall be the chief executive officer of the
corporation and, subject to the control of the directors, shall
have general charge and supervision of the affairs of the
corporation, including but not limited to being signatory of the
corporate checking account.

\end{asec}
\begin{asec}[Treasurer]

The treasurer shall be the chief financial officer and the chief
accounting officer of the corporation, who shall be in charge of
its financial affairs, and keep accurate records thereof.  The
treasurer may have such other duties and \index{power}powers as designated by
the directors, including but not limited to being signatory of the
corporate checking account.


\end{asec}

\begin{asec}[Clerk]

The clerk shall keep and maintain corporation files, including
archives of the directors \index{meeting}meetings notes, which shall be kept at
the corporation's principle office in the state where the
corporation is incorporated.  Such records shall also include
corporate articles of organization, bylaws, and the names and
addresses of current directors.

\end{asec}

\begin{asec}[Other Officers and Agents]

The directors may appoint such officers and agents as it may deem
advisable, who shall hold their offices for such terms and shall
exercise such \index{power}power and perform such duties as shall be determined
by the directors.

\end{asec}

\begin{asec}[Resignation]

An officer may resign at any time for health or personal reasons.

\end{asec}


\begin{asec}[Vacancies]

If the office of any officer becomes vacant, the directors may
elect a successor, who shall serve until the next annual \index{meeting}meeting
at which point he or she could be elected to another term, or a
different officer elected.

\end{asec}

\end{art}

\begin{art}[Execution of Papers] ~


\begin{asec}[Instruments]

All deeds, leases, transfers, contracts, 
bonds, notes, checks, drafts, and other obligations 
made, accepted or endorsed by the corporation must 
be signed by the president or the treasurer.  Any recordable 
instrument purporting to affect an interest in real 
estate, executed in the name of the corporation by 
two of its officers, of whom one is the president and 
the other is the treasurer, shall be binding of the 
corporation in favor of a purchaser or other person 
relying in good faith upon such instrument notwithstanding 
any inconsistent provisions of the articles of organization, 
bylaws, resolutions, or decisions of the corporation.

\end{asec}
\end{art}


\begin{art}[Personal Liability] ~


The directors and officers of the corporation shall 
not be personally liable for any debt, liability, or 
obligation of the corporation.  All persons, corporations, 
or other entities extending credit to, contracting 
with, or having any claim against, the corporation, 
may look only to the funds and property of the corporation 
for the payment of any such contract or claim or for 
the payment of any debt, damages, judgment or decree, 
or of any money that may otherwise become due or payable 
to them from the corporation.

\end{art}

\begin{art}[Disbursement of Earnings and Assets] ~
\begin{asec}[Net Earnings]

No part of the net earnings of the corporation shall inure to the
benefit of, or be distributable to its members, officers, or other
private persons, except that the corporation shall be empowered
and authorized to pay reasonable compensation for services
rendered and to make payments and distributions in furtherance of
the purposes set forth in Article 2 hereof.

\end{asec}
\begin{asec}[Dissolution]

%Here replace State Law Code c.123 with appropriate references.

Upon the dissolution of the corporation, assets shall be
distributed for one or more exempt purposes within the meaning of
section 501 (c) (3) of the Internal Revenue Code, or corresponding
section of any future federal tax code, or in the manner
prescribed by State Law Code, chapter 123, section 456, or
corresponding chapter of any future state statue.
\end{asec}
\end{art}

\begin{art}[Amendments] ~


These bylaws may be altered, amended, or repealed in 
whole or in part by consent of the directors at two 
consecutive annual directors \index{meeting}meetings.
\end{art}
\clearpage

\begin{art}[Application of State Law Code c.123] ~

%Here replace State Law Code c.123 with appropriate references.

To the extent that any provision of these bylaws is inconsistent
with State Law Code c.123, it is the intent of these bylaws that
c.123 shall supersede these bylaws and apply.  To the extent that
these bylaws do not make provision for any corporate action, and
c.123 does make such provision, c.123 shall apply.

\end{art}
