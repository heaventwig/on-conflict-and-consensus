%Spelling checked 2004-04-06
\index{meeting}Meetings can often be a time when some people experience feelings
of frustration or confusion. There is always room for improvement
in the \index{structure}structure of the process and/or in the dynamics of the
group. Often, there is no time to talk directly about group
interaction during the \index{meeting}meeting. Reserve time at the end of the
\index{meeting}meeting to allow some of these issues and feelings to be expressed.

\index{evaluation}Evaluation is very useful when using consensus. It is worth the
time. \index{evaluation}Evaluations need not take long, five to ten minutes is often
enough. It is not a discussion, nor is it an opportunity to
comment on each other's statements. Do not reopen discussion on an
\index{agenda}agenda item. \index{evaluation}Evaluation is a special time to listen to each other
and learn about each other. Think about how the group interacts and
how to improve the process.

Be sure to include the \index{evaluation}evaluation comments in the notes of the
\index{meeting}meeting. This is important for two reasons. Over time, if the same
\index{evaluation}evaluation comments are made again and again, this is an indication
that the issue behind the comments needs to be addressed. This can
be accomplished by placing this issue on the \index{agenda}agenda for the next
\index{meeting}meeting. Also, when looking back at notes from \index{meeting}meetings long ago,
\index{evaluation}evaluation comments can often reveal a great deal about what
actually happened, beyond what decisions were made and
reports given. They give a glimpse into complex interpersonal
dynamics.

\section{Purpose of \index{evaluation}evaluation}

\index{evaluation}Evaluation provides a forum to address procedural flaws,
inappropriate behavior, facilitation problems, logistical
difficulties, overall tone, etc. \index{evaluation}Evaluation is not a time to reopen
discussion, make decisions or attempt to resolve problems, but
rather, to make statements, express feelings, highlight problems,
and suggest solutions in a spirit of \index{cooperation}cooperation and trust. To help
foster communication, it is better if each criticism is coupled
with a specific suggestion for improvement. Also, always speak for
oneself. Do not attempt to represent anyone else.

Encourage everyone who participated in the \index{meeting}meeting to take part in
the \index{evaluation}evaluation. Make comments on what worked and what did
not. Expect differing opinions. It is generally not useful to
repeat other's comments. \index{evaluation}Evaluations prepare the group for better
future \index{meeting}meetings. When the process works well, or the group responds
supportively in a difficult situation, or the \index{facilitator}facilitator does an
especially good job, note it, and appreciate work well done.

Do not attempt to force \index{evaluation}evaluation. This will cause superficial or
irrelevant comments. On the other hand, do not allow \index{evaluation}evaluations to
run on. Be sure to take each comment seriously and make an attempt,
at a later time, to resolve or implement them. Individuals who feel
their suggestions are ignored or dis\index{respect}respected will lose trust and
interest in the group.

For gatherings, conferences, conventions or large \index{meeting}meetings, the
group might consider having short \index{evaluation}evaluations after each section,
in addition to the one at the end of the event. Distinct aspects on
which the group might focus include: the process itself, a specific
\index{role}role, a particular technique, fears and feelings, group dynamics,
etc.

At large \index{meeting}meetings, written \index{evaluation}evaluations provide a means for
everyone to respond and record comments and suggestions which might
otherwise be lost. Some people feel more comfortable writing their
\index{evaluation}evaluations rather than saying them. Plan the questions well,
stressing what was learned, what was valuable, and what could have
been better and how. An \index{evaluation}evaluation committee allows an opportunity
for the presenters, \index{facilitator}facilitators, and/or coordinators to get
together after the \index{meeting}meeting to review \index{evaluation}evaluation comments, consider
suggestions for improvement, and possibly prepare an \index{evaluation}evaluation
report.

Review and \index{evaluation}evaluation bring a sense of completion to the \index{meeting}meeting. A good \index{evaluation}evaluation will pull the experience together, remind everyone of the group's \index{unity of purpose}unity of purpose, and provide an opportunity for closing comments.

\section{Uses of \index{evaluation}evaluation}

There are at least ten ways in which \index{evaluation}evaluation helps improve \index{meeting}meetings. \index{evaluation}Evaluations:

\squishitemize%\begin{itemize}
\item Improve the process by analysis of what happened, why it
happened, and how it might be improved
\item Examine how certain attitudes and statements might have
  caused various problems and encourage special care to prevent
  them from recurring
\item Foster a greater understanding of group dynamics and encourage a method of group learning or learning from each other
\item Allow the free expression of feelings
\item Expose unconscious behavior or attitudes which interfere
  with the process
\item Encourage the sharing of observations and acknowledge
  associations with society
\item Check the usefulness and effectiveness of \index{techniques}techniques and
  procedures
\item Acknowledge good work and give appreciation to each other
\item Reflect on the goals set for the \index{meeting}meeting and whether they
  were attained
\item Examine various \index{role}roles, suggest ways to improve them, and
  create new ones as needed
\item Provide an overall sense of completion and closure to the
  \index{meeting}meeting
\squishend%\end{itemize}

\section{Types of \index{evaluation}evaluation questions}

It is necessary to be aware of the \emph{way} in which questions
are asked during \index{evaluation}evaluation. The specific wording can control the
scope and focus of consideration and affect the level of
\index{participation}participation. It can cause responses which focus on what was good
and bad, or right and wrong, rather than on what worked and what
needed improvement. Focus on learning and growing. Avoid
blaming. Encourage diverse opinions.

\subsection*{Some sample questions for an \index{evaluation}evaluation:}



\squishitemize%\begin{itemize}
\item Were members uninterested or bored with the \index{agenda}agenda, reports,
  or discussion?
\item Did members withdraw or feel isolated?
\item Is attendance low? If so, why?
\item Are people arriving late or leaving early? If so, why?
\item How was the overall tone or atmosphere?
\item Was there an appropriate use of resources?
\item Were the logistics (such as date, time, or location)
  acceptable?
\item What was the most important experience of the event?
\item What was the least important experience of the event?
\item What was the high point? What was the low point?
\item What did you learn?
\item What expectations did you have at the beginning and to what
  degree were they met? How did they change?
\item What goals did you have and to what degree were they
  accomplished?
\item What worked well? Why?
\item What did not work so well? How could it have been improved?
\item What else would you suggest be changed or improved, and how?
\item What was overlooked or left out?
\squishend%\end{itemize}
