%Spelling checked 2004-04-01
There are many ways to make decisions. Sometimes, the most
efficient way to make decisions is to just let the manager
(or CEO, or dictator) make them. However, efficiency is not the
only criterion. When choosing a \index{decisionmaking}decisionmaking method, one needs to
ask two questions. Is it a fair process? Does it produce good
solutions?

To judge the process, consider the following: Does the \index{meeting}meeting
flow smoothly? Is the discussion kept to the point? Does it take
too long to make each decision? Does the leadership determine the
outcome of the discussion? Are some people overlooked?

To judge the quality of the end result, the \emph{decision,}
consider: Are the people making the decision, and all those
affected, satisfied with the result? To what degree is the intent
of the original \emph{proposal} accomplished? Are the underlying
issues addressed? Is there an appropriate use of resources? Would
the group make the same decision again?

Autocracy can work, but the idea of a benevolent dictator is just
a dream. We believe that it is inherently better to involve every
person who is affected by the decision in the \index{decisionmaking}decisionmaking
process. This is true for several reasons. The decision would
reflect the will of the entire group, not just the leadership. The
people who carry out the plans will be more satisfied with their
work. And, as the old adage goes, two heads are better than one.

This book presents a particular model for \index{decisionmaking}decisionmaking we call
Formal Consensus. Formal Consensus has a clearly defined
\index{structure}structure. It requires a \index{commitment}commitment to active \index{cooperation}cooperation,
disciplined speaking and listening, and \index{respect}respect for the
contributions of every member. Likewise, every person has the
responsibility to actively participate as a creative
individual within the \index{structure}structure.

Avoidance, denial, and repression of conflict is common during
\index{meeting}meetings.  Therefore, using Formal Consensus might not be easy at
first. Unresolved conflict from previous experiences could come
rushing forth and make the process difficult, if not
impossible. Practice and discipline, however, will smooth the
process. The benefit of everyone's \index{participation}participation and \index{cooperation}cooperation is
worth the struggle it may initially take to ensure that all voices
are heard.

It is often said that consensus is time-consuming and
difficult. Making complex, difficult decisions is time-consuming,
no matter what the process. Many different methods can be
efficient, if every participant shares a common understanding of
the rules of the game. Like any process, Formal Consensus can be
inefficient if a group does not first assent to follow a
particular \index{structure}structure.

This book codifies a formal \index{structure}structure for \index{decisionmaking}decisionmaking. It is
hoped that the relationship between this book and Formal Consensus
would be similar to the relationship between Robert's Rules of
Order and Parliamentary Procedure.

Methods of \index{decisionmaking}decisionmaking can be seen on a continuum with one
person having total authority on one end and sharing \index{power}power
and responsibility on the other.

The level of \index{participation}participation increases along this \index{decisionmaking}decisionmaking
continuum. Oligarchies and autocracies offer no \index{participation}participation to
many of those who are directly affected. Representative, majority
rule, and consensus democracies involve everybody, to different
degrees.

\section{Group Dynamics}

A group, by definition, is a number of individuals having some
unifying relationship. The group dynamic created by consensus
process is completely different from that of Parliamentary
Procedure, from start to finish. It is based on different values
and uses a different language, a different \index{structure}structure, and many
different \index{techniques}techniques, although some \index{techniques}techniques overlap. It
might be helpful to explain some broad concepts about group
dynamics and consensus.

\subsection*{Conflict}

While \index{decisionmaking}decisionmaking is as much about conflict as it is about
agreement, Formal Consensus works best in an atmosphere in which
conflict is encouraged, supported, and resolved cooperatively with
\index{respect}respect, nonviolence, and creativity.  Conflict is desirable. It
is not something to be avoided, dismissed, diminished, or denied.

\subsection*{Majority Rule and Competition}

Generally speaking, when a group votes using majority rule or
Parliamentary Procedure, a competitive dynamic is created within
the group because it is being asked to choose between two (or more)
possibilities. It is just as acceptable to attack and diminish
another's point of view as it is to promote and endorse your own
ideas. Often, voting occurs before one side reveals anything about
itself, but spends time solely attacking the opponent! In this
adversarial environment, one's ideas are owned and often defended
in the face of improvements.

\subsection*{Consensus and \index{cooperation}Cooperation}

Consensus process, on the other hand, creates a cooperative
dynamic. Only one proposal is considered at a time. Everyone works
together to make it the best possible decision for the group. Any
concerns are raised and resolved, sometimes one by one, until all
voices are heard. Since proposals are no longer the property of the
presenter, a solution can be created more cooperatively.

\subsection*{Proposals}

In the consensus process, only proposals which intend to
accomplish the common purpose are considered. During discussion of
a proposal, everyone works to improve the proposal to make it the
best decision for the group. All proposals are adopted unless the
group decides it is contrary to the best interests of the group.

\section[Characteristics]{Characteristics\\ of Formal Consensus}

Before a group decides to use Formal Consensus, it must honestly
assess its ability to honor the principles described in Chapter
Three. If the principles described in this book are not already
present or if the group is not willing to work to create them, then
Formal Consensus will not be possible. Any group which wants to
adopt Formal Consensus needs to give considerable attention to the
underlying principles which support consensus and help the process
operate smoothly. This is not to say each and every one of the
principles described herein must be adopted by every group, or
that each group cannot add its own principles specific to its
goals, but rather, each group must be very clear about the
foundation of principles or common purposes they choose before they
attempt the Formal Consensus \index{decisionmaking}decisionmaking process.

\subsection*{Formal Consensus is the least violent \index{decisionmaking}decisionmaking process.} % (fold)
\label{sub:formal_consensus_is_the_least_violent_decisionmakingdecisionmaking_process_}
Traditional nonviolence theory holds that the use of \index{power}power to
dominate is violent and undesirable. Nonviolence expects people to
use their \index{power}power to persuade without deception, coercion, or malice,
using truth, creativity, logic, \index{respect}respect, and love. Majority rule
voting process and Parliamentary Procedure both accept, and even
encourage, the use of \index{power}power to dominate others. The goal is the
winning of the vote, often regardless of another choice which
might be in the best interest of the \index{whole group}whole group. The will of the
majority supersedes the concerns and desires of the minority. This
is inherently violent. Consensus strives to take into account
everyone's concerns and resolve them before any decision is
made. Most importantly, this process encourages an environment in
which everyone is \index{respect}respected and all contributions are valued.

% subsection formal_consensus_is_the_least_violent_decisionmakingdecisionmaking_process_ (end)

\subsection*{Formal Consensus is the most democratic \index{decisionmaking}decisionmaking process.}

Groups which desire to involve as many people as possible need to
use an inclusive process. To attract and involve large numbers, it
is important that the process encourages \index{participation}participation, allows
\index{equal access to power}equal access to \index{power}power, develops \index{cooperation}cooperation, promotes empowerment,
and creates a sense of individual responsibility for the group's
actions. All of these are cornerstones of Formal Consensus. The
goal of consensus is not the selection of several options, but the
development of one decision which is the best for the whole
group. It is synthesis and evolution, not competition and
attrition.

\subsection*{Formal Consensus is based on the principles of the group.}

Although every individual must consent to a decision before it is
adopted, if there are any objections, it is not the choice of the
individual alone to determine if an objection prevents the proposal
from being adopted. Every objection or concern must first be
presented before the group and either resolved or validated. A
valid objection is one in keeping with all previous decisions of
the group and based upon the commonly-held principles or foundation
adopted by the group. The objection must not only address the
concerns of the individual, but it must also be in the best
interest of the group as a whole. If the objection is not based
upon the foundation, or is in contradiction with a prior decision,
it is not valid for the group, and therefore, out of order.

\subsection*{Formal Consensus is desirable in larger groups.}

If the \index{structure}structure is vague, decisions can be difficult to
achieve. They will become increasingly more difficult in larger
groups. Formal Consensus is designed for large groups. It is a
highly \index{structure}structured model. It has guidelines and formats for managing
\index{meeting}meetings, facilitating discussions, resolving conflict, and
reaching decisions. Smaller groups may need less \index{structure}structure, so
they may choose from the many \index{techniques}techniques and \index{role}roles suggested in
this book.

\subsection*{Formal consensus works better when more people participate.}
\index{participation}
Consensus is more than the sum total of ideas of the individuals
in the group. During discussion, ideas build one upon the next,
generating new ideas, until the best decision emerges. This
dynamic is called the creative interplay of ideas. Creativity plays
a major part as everyone strives to discover what is best for the
group. The more people involved in this cooperative process, the
more ideas and possibilities are generated. Consensus works best
with everyone participating. (This assumes, of course, that
everyone in the group is trained in Formal Consensus and is
actively using it.)

\subsection*{Formal Consensus is not inherently time-consuming.}
Decisions are not an end in themselves. \index{decisionmaking}Decisionmaking is a
process which starts with an idea and ends with the actual
implementation of the decision. While it may be true in an
autocratic process that decisions can be made quickly, the actual
implementation will take time. When one person or a \index{small group}small group of
people makes a decision for a larger group, the decision not only
has to be communicated to the others, but it also has to be
acceptable to them or its implementation will need to be forced
upon them. This will certainly take time, perhaps a considerable
amount of time. On the other hand, if everyone participates in the
\index{decisionmaking}decisionmaking, the decision does not need to be communicated and
its implementation does not need to be forced upon the
participants. The decision may take longer to make, but once it is
made, implementation can happen in a timely manner. The amount of
time a decision takes to make from start to finish is not a factor
of the process used; rather, it is a factor of the complexity of
the proposal itself. An easy decision takes less time than a
difficult, complex decision, regardless of the process used or the
number of people involved. Of course, Formal Consensus works
better if one practices \index{patience}patience, but any process is improved with
a generous amount of \index{patience}patience.

\subsection*{Formal Consensus cannot be secretly disrupted.}

This may not be an issue for some groups, but many people know
that the state actively surveils, infiltrates, and disrupts
nonviolent domestic political and religious groups. To counteract
anti-democratic tactics by the state, a group would need to develop
and encourage a \index{decisionmaking}decisionmaking process which could not be covertly
controlled or manipulated. Formal Consensus, if practiced as
described in this book, is just such a process. Since the
assumption is one of \index{cooperation}cooperation and \index{good will}good will, it is always
appropriate to ask for an explanation of how and why someone's
actions are in the best interest of the group. Disruptive behavior
must not be tolerated. While it is true this process cannot
prevent openly disruptive behavior, the point is to prevent covert
disruption, hidden \index{agenda}agenda, and malicious manipulation of the
process. Any group for which infiltration is a threat ought to
consider the process outlined in this book if it wishes to remain
open, democratic, and productive.
