%Spelling checked 2004-04-01
Decisions are adopted when all participants consent to the result
of discussion about the original proposal. People who do not agree
with a proposal are responsible for expressing their concerns. No
decision is adopted until there is resolution of every
concern. When concerns remain after discussion, individuals can
agree to disagree by acknowledging that they have
unresolved concerns, but consent to the proposal anyway and allow
it to be adopted.  Therefore, reaching consensus does not assume
that everyone must be in complete agreement, a highly unlikely
situation in a group of intelligent, creative individuals.

Consensus is becoming popular as a democratic form of
\index{decisionmaking}decisionmaking.  It is a process which requires an environment in
which all contributions are valued and \index{participation}participation is
encouraged. There are, however, few organizations which use a model
of consensus which is specific, consistent, and efficient.  Often,
the consensus process is informal, vague, and very
inconsistent.  This happens when the consensus process is not based
upon a solid foundation and the \index{structure}structure is unknown or
nonexistent. To develop a more formal type of consensus process,
any organization must define the commonly held principles which
form the foundation of the group's work and intentionally choose
the type of \index{structure}structure within which the process is built.

This book contains the building materials for just such a
process. Included is a description of the principles from which a
foundation is created, the flowchart and levels of \index{structure}structure which
are the frame for the process, and the other materials needed for
designing a variety of processes which can be customized to fit the
needs of the organization.

\section[The Structure]{The \index{structure}Structure\\ of Formal Consensus}

Many groups regularly use diverse discussion \index{techniques}techniques learned
from practitioners in the field of conflict resolution. Although
this book does include several \index{techniques}techniques, the book is about a
\emph{\index{structure}structure} called Formal Consensus.  This \index{structure}structure creates
a separation between the \emph{\index{identification}identification} and the
\emph{resolution} of concerns. Perhaps, if everybody in the group
has no trouble saying what they think, they won't need this
\index{structure}structure. This predictable \index{structure}structure provides opportunities to
those who don't feel empowered to participate.

Formal Consensus is presented in levels or cycles. In the first
level, the idea is to allow everyone to express their perspective,
including concerns, but group time is not spent on resolving
problems. In the second level the group focuses its attention on
identifying concerns, still not resolving them. This requires
discipline. Reactive comments, even funny ones, and resolutions,
even good ones, can suppress the creative ideas of others. Not
until the third level does the \index{structure}structure allow for exploring
resolutions.

Each level has a different scope and focus. At the first level,
the scope is broad, allowing the discussion to consider the
philosophical and political implications as well as the general
merits and drawbacks and other relevant information. The only focus
is on the proposal as a whole. Some decisions can be reached after
discussion at the first level. At the second level, the scope of
the discussion is limited to the concerns. They are identified and
publicly listed, which enables everyone to get an overall picture
of the concerns. The focus of attention is on identifying the
body of concerns and grouping similar ones. At the third level, the
scope is very narrow. The focus of discussion is limited to a
single unresolved concern until it is resolved.

\section[The Flow]{The Flow\\ of Formal Consensus}

In an ideal situation, every proposal would be submitted in
writing and briefly introduced the first time it appears on the
\index{agenda}agenda. At the next \index{meeting}meeting, after everyone has had enough time to
read it and carefully consider any concerns, the discussion would
begin in earnest. Often, it would not be until the third \index{meeting}meeting
that a decision is made. Of course, this depends upon how many
proposals are on the table and the urgency of the decision.

\subsubsection*{Clarify the Process}

The \index{facilitator}facilitator introduces the person presenting the proposal and
gives a short update on any previous action on it. It is very
important for the \index{facilitator}facilitator to explain the process which brought
this proposal to the \index{meeting}meeting, and to describe the process that will
be followed to move the group through the proposal to consensus. It
is the \index{facilitator}facilitator's job to make sure that every participant
clearly understands the \index{structure}structure and the discussion \index{techniques}techniques
being employed while the \index{meeting}meeting is in progress.

\subsubsection*{Present Proposal or Issue}

When possible and appropriate, proposals ought to be prepared in
writing and distributed well in advance of the \index{meeting}meeting in which a
decision is required. This encourages prior discussion and
consideration, helps the presenter anticipate concerns, minimizes
surprises, and involves everyone in creating the proposal. (If the
necessary groundwork has not been done, the wisest choice might be
to send the proposal to committee. Proposal writing is difficult to
accomplish in a large group. The committee would develop
the proposal for consideration at a later time.) The presenter
reads the written proposal aloud, provides background information,
and states clearly its benefits and reasons for adoption, including
addressing any existing concerns.

\subsubsection*{Questions Which Clarify the Presentation}
\index{clarifying questions}
Questions are strictly limited by the \index{facilitator}facilitator to those which
seek greater comprehension of the proposal as presented. Everyone
deserves the opportunity to fully understand what is being asked
of the group before discussion begins. This is not a time for
comments or concerns. If there are only a few questions, they can
be answered one at a time by the person presenting the
proposal. If there are many, a useful technique is hearing all the
questions first, then answering them together. After answering all
\index{clarifying questions}clarifying questions, the group begins discussion.


%\begin{figure}
%\begin{center}
%\includegraphics[width=\textwidth]{flowchart}
%\end{center}
%\end{figure}

\subsection*{Level One: Broad Open Discussion}
\subsubsection*{General Discussion}

Discussion at this level ought to be the broadest in scope. Try to
encourage comments which take the whole proposal into account;
i.e., why it is a good idea, or general problems which need to be
addressed. Discussion at this level often has a philosophical or
principled tone, purposely addressing how this proposal might
affect the group in the long run or what kind of precedent it might
create, etc. It helps every proposal to be discussed in this way,
before the group engages in resolving particular concerns. Do not
allow one concern to become the focus of the discussion. When
particular concerns are raised, make note of them but encourage the
discussion to move back to the proposal as a whole. Encourage the
creative interplay of comments and ideas. Allow for the addition of
any relevant factual information. For those who might at first feel
opposed to the proposal, this discussion is consideration of why it
might be good for the group in the broadest sense. Their initial
concerns might, in fact, be of general concern to the whole
group. And, for those who initially support the proposal, this is a
time to think about the proposal broadly and some of the general
problems. If there seems to be general approval of the proposal,
the \index{facilitator}facilitator, or someone recognized to speak, can request a call
for consensus.

\subsubsection*{Call for Consensus}

The \index{facilitator}facilitator asks, ``Are there any unresolved concerns?'' or ``Are
there any concerns remaining?'' After a period of \index{silence}silence, if no
additional concerns are raised, the \index{facilitator}facilitator declares that
consensus is reached and the proposal is read for the record. The
length of \index{silence}silence ought to be directly related to the degree of
difficulty in reaching consensus; an easy decision requires a short
\index{silence}silence, a difficult decision requires a longer \index{silence}silence.  This
encourages everyone to be at peace in accepting the consensus
before moving on to other business. At this point, the \index{facilitator}facilitator
assigns task responsibilities or sends the decision to a committee
for implementation.  It is important to note that the question is
not ``Is there consensus?'' or ``Does everyone agree?'' These
questions do not encourage an environment in which all concerns can
be expressed. If some people have a concern, but are shy or
intimidated by a strong showing of support for a proposal, the
question ``Are there any unresolved concerns?'' speaks directly to
them and provides an opportunity for them to speak. Any concerns
for which someone stands aside are listed with the proposal and
become a part of it.
%\vfill

\subsection*{Level Two: Identify Concerns}

\subsubsection*{List Any Concerns}

At the beginning of the next level, a discussion technique called \index{brainstorming}brainstorming (see section \ref{sec:brainstorming}) is used so that concerns can be identified and written down publicly by the scribe and for the record by the \index{notetaker}notetaker. Be sure the scribe is as accurate as possible by checking with the person who voiced the concern before moving on. This is not a time to attempt to resolve concerns or determine their validity. That would stifle free expression of concerns. At this point, only concerns are to be expressed, reasonable or unreasonable, well thought out or vague feelings. The \index{facilitator}facilitator wants to interrupt any comments which attempt to defend the proposal, resolve the concerns, judge the value of the concerns, or in any way deny or dismiss another's feelings of doubt or concern. Sometimes simply allowing a concern to be expressed and written down helps resolve it. After most concerns have been listed, allow the group a moment to reflect on them as a whole.

\subsubsection*{Aggregate Related Concerns}

At this point, the focus is on identifying patterns and relationships between concerns. This short exercise must not be allowed to focus upon or resolve any particular concern.

\subsection*{Level Three: Resolve Concerns}

\subsubsection*{Resolve Groups of Related Concerns}

Often, related concerns can be resolved as a group.

\subsubsection*{Call for Consensus} % (fold)
\label{ssub:call_for_consensus}
If most of the concerns seem to have been resolved, call for consensus in the manner described earlier. If some concerns have not been resolved at this time, then a more focused discussion is needed.
% subsubsection call_for_consensus (end)

\subsubsection[Restate Remaining Concerns]{Restate Remaining Concerns (One at a Time)}

Return to the list. The \index{facilitator}facilitator checks each one with the group and removes ones which have been resolved or are, for any reason, no longer of concern. Each remaining concern is restated clearly and concisely and addressed one at a time. Sometimes new concerns are raised which need to be added to the list. Every individual is responsible for honestly expressing concerns as they think of them. It is not appropriate to hold back a concern and spring it upon the group late in the process. This undermines trust and limits the group's ability to adequately discuss the concern in its relation to other concerns.

\subsubsection*{Questions Which Clarify the Concern}

The \index{facilitator}facilitator asks for any questions or comments which would further clarify the concern so everyone clearly understands it \emph{before} discussion starts.

\subsubsection*{Discussion Limited to Resolving One Concern}

Use as many creative \index{group discussion techniques}group discussion \index{techniques}techniques as needed to
facilitate a resolution for each concern. Keep the discussion
focused upon the particular concern until every suggestion has been
offered. If no new ideas are coming forward and the concern cannot
be resolved, or if the time allotted for this item has been
entirely used, move to one of the closing options described below.

\subsubsection*{Call for Consensus}

Repeat this process until all concerns have been resolved. At this point, the group should be at consensus, but it would be appropriate to call for consensus anyway just to be sure no concern has been overlooked.

\subsection*{Closing Options}
\subsubsection*{Send to Committee}
If a decision on the proposal can wait until the \index{whole group}whole group meets again, then send the proposal to a committee which can clarify the concerns and bring new, creative resolutions for consideration by the group. It is a good idea to include on the committee representatives of all the major concerns, as well as those most supportive of the proposal so they can work out solutions in a less formal setting. Sometimes, if the decision is needed before the next \index{meeting}meeting, a smaller group can be empowered to make the decision for the larger group, but again, this committee should include all points of view. Choose this option only if it is absolutely necessary and the \index{whole group}whole group consents.

\subsubsection*{\index{stand aside}Stand Aside (decision adopted with unresolved concerns listed)}

When a concern has been fully discussed and cannot be resolved, it is appropriate for the \index{facilitator}facilitator to ask those persons with this concern if they are willing to \index{stand aside}stand aside; that is, acknowledge that the concern still exists, but allow the proposal to be adopted. It is very important for the \index{whole group}whole group to understand that this unresolved concern is then written down with the proposal in the record and, in essence, becomes a part of the decision. This concern can be raised again and deserves more discussion time as it has not yet been resolved. In contrast, a concern which has been resolved in past discussion does not deserve additional discussion, unless something new has developed. Filibustering is not appropriate in Formal Consensus.

\subsubsection*{Declare \index{block}Block}

After having spent the allotted \index{agenda}agenda time moving through the
three levels of discussion trying to achieve consensus and concerns
remain which are unresolved, the \index{facilitator}facilitator is obligated to
declare that consensus cannot be reached at this \index{meeting}meeting, that the
proposal is blocked, and move on to the next \index{agenda}agenda item.

%\clearpage
\begin{figure}[htb]
\caption{The Formal Consensus Flow Chart}\label{fig:flowchart}
\begin{center}
%\fbox{
\includegraphics%[width=1\textwidth]
[width=0.95\textwidth]{consensus-flowchart-insert}%}  %Notice how this line is different
\end{center}
\end{figure}

\section[The Rules]{The Rules\\ of Formal Consensus}

The guidelines and \index{techniques}techniques in this book are flexible and meant
to be modified. Some of the guidelines, however, seem almost always
to be true. These are the Rules of Formal Consensus: 

\begin{enumerate}
\item Once a decision has been adopted by consensus, it cannot be
  changed without reaching a new consensus. If a new consensus
  cannot be reached, the old decision stands.

\item In general, only one person has permission to speak at any
  moment. The person with permission to speak is determined by the
  group discussion technique in use and/or the \index{facilitator}facilitator. (The
  \index{role}role of \index{peacekeeper}Peacekeeper is exempt from this rule.)

\item All structural decisions (i.e., which \index{role}roles to use, who
  fills each \index{role}role, and which facilitation technique and/or group
  discussion technique to use) are adopted by consensus without
  debate. Any objection automatically causes a new selection to be
  made. If a \index{role}role cannot be filled without objection, the group
  proceeds without that \index{role}role being filled. If much time is spent
  trying to fill \index{role}roles or find acceptable \index{techniques}techniques, then the
  group needs a discussion about the \index{unity of purpose}unity of purpose of this
  group and why it is having this problem, a discussion which must
  be put on the \index{agenda}agenda for the next \index{meeting}meeting, if not held
  immediately.

\item All content decisions (i.e., the \index{agenda}\index{agenda contract}agenda contract, committee
  reports, proposals, etc.) are adopted by consensus after
  discussion. Every content decision must be openly discussed
  before it can be tested for consensus.

\item A concern must be based upon the principles of the group to
  justify a \index{block}block to consensus.

\item Every \index{meeting}meeting which uses Formal Consensus must have an
  \index{evaluation}evaluation.
\end{enumerate}
