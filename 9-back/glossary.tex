\begin{description}
\item [\index{agenda}\index{agenda contract}agenda contract] The \index{agenda}\index{agenda contract}agenda contract is made when the \index{agenda}agenda is reviewed and accepted. This agreement includes the items on the \index{agenda}agenda, the order in which they are considered, and the time allotted to each.  Unless the \index{whole group}whole group agrees to change the \index{agenda}agenda, the \index{facilitator}facilitator is obligated to keep to the contract.  The decision to change the \index{agenda}agenda must be a consensus, with little or no discussion.
\item [agreement] Complete agreement, with no unresolved concerns.
\item [\index{block}block] If the allotted \index{agenda}agenda time has been spent trying to achieve consensus, and unresolved legitimate concerns remain, the proposal may be considered blocked, or not able to be adopted at this \index{meeting}meeting.
\item [concern]
A point of departure or disagreement with a proposal.
\item [conflict]
The expression of disagreement, which brings into focus diverse viewpoints, and provides the opportunity to explore their strengths and weaknesses.
\item [consensus]
A \index{decisionmaking}decisionmaking process whereby decisions are reached when all members present consent to a proposal. This process does not assume everyone must be in complete agreement. When differences remain after discussion, individuals can agree to disagree, that is, give their consent by standing aside, and allow the proposal to be accepted by the group.
\item [consent]
Acceptance of the proposal, not necessarily agreement. Individuals are responsible for expressing their ideas, concerns and objections. \index{silence}Silence, in response to a call for consensus, signifies consent. \index{silence}Silence is not complete agreement; it is acceptance of the proposal.
\item [decision]
The end product of an idea that started as a proposal and evolved to become a plan of action accepted by the \index{whole group}whole group.
\item [\index{evaluation}evaluation]
A group analysis at the end of a \index{meeting}meeting about interpersonal dynamics during \index{decisionmaking}decisionmaking.  This is a time to allow feelings to be expressed, with the goal of improving the functioning of future \index{meeting}meetings. It is not a discussion or debate, nor should anyone comment on another's \index{evaluation}evaluation.
\item [\index{meeting}meeting]
An occasion in which people come together and, in an orderly way, make decisions.
\item [methods of \index{decisionmaking}decisionmaking] ~
\begin{description}
\item [autocracy]
one person makes the decisions for everyone
\item [oligarchy]
a few people make the decisions for everyone
\item [representative democracy]
a few people are\\ elected to make the decisions for everyone
\item [majority rule democracy]
the majority makes the decisions for everyone
\item [consensus]
everyone makes the decisions for everyone

\end{description}
\item [proposal]
A written plan that some members of a group present to the \index{whole group}whole group for discussion and acceptance.
\item [\index{stand aside}stand aside] To agree to disagree, to be willing to let a proposal be adopted despite unresolved concerns.
\end{description}
 