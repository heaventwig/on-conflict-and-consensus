\section{Facilitation Techniques}
\index{facilitator techniques}
There are a great many \index{techniques}techniques to assist the \index{facilitator}facilitator in
managing the \index{agenda}agenda and group dynamics. The following are just a
few of the more common and frequently used \index{techniques}techniques available to
the \index{facilitator}facilitator. Be creative and adaptive. Different situations
require different \index{techniques}techniques.With experience will come an
understanding of how they affect group dynamics and when is the
best time to use them.

\subsection{\index{equalizing participation}Equalizing \index{participation}Participation}

The \index{facilitator}facilitator is responsible for the fair distribution of
attention during \index{meeting}meetings. \index{facilitator}Facilitators call the attention of the
group to one speaker at a time. The grammar school method is the
most common technique for choosing the next speaker. The
\index{facilitator}facilitator recognizes each person in the order in which hands are
raised. Often, inequities occur because the attention is dominated
by an individual or class of individuals. This can occur because of
socialized behavioral problems such as racism, sexism, or the
like, or internal dynamics such as experience, seniority, fear,
shyness, dis\index{respect}respect, ignorance of the process, etc. Inequities can
be corrected in many creative ways. For example, if men are
speaking more often than women, the \index{facilitator}facilitator can suggest a pause
after each speaker, the women counting to five before speaking, the
men counting to ten. In controversial situations, the
\index{facilitator}facilitator can request that three speakers speak for the proposal,
and three speak against it. If the group would like to avoid having
the \index{facilitator}facilitator select who speaks next, the group can self-select
by asking the last speaker to pass an object, a talking stick, to
the next. Even more challenging, have each speaker stand before
speaking, and begin when there is only one person standing. These
are only a handful of the many possible problems and solutions that
exist. Be creative. Invent your own.

\subsection{\index{listing}Listing}

To help the discussion flow more smoothly, those who want to speak
can silently signal the \index{facilitator}facilitator, who would add the person's
name to a list of those wishing to speak, and call on them in that
order.

\subsection{\index{stacking}Stacking}

If many people want to speak at the same time, it is useful to ask
all those who would like to speak to raise their hands. Have them
count off, and then have them speak in that order. At the end of
the stack, the \index{facilitator}facilitator might call for another stack or try
another technique.

\subsection{\index{pacing}Pacing}

The pace or flow of the \index{meeting}meeting is the responsibility of the
\index{facilitator}facilitator.  If the atmosphere starts to become tense, choose
\index{techniques}techniques which encourage balance and \index{cooperation}cooperation. If the \index{meeting}meeting
is going slowly and people are becoming restless, suggest a stretch
or rearrange the \index{agenda}agenda.

\subsection{\index{checking the process}Checking the Process}

If the flow of the \index{meeting}meeting is breaking down or if one person or
\index{small group}small group seems to be dominating, anyone can call into question
the technique being used and suggest an alternative.

\subsection{\index{silence}Silence}

If the pace is too fast, if energies and tensions are high, if
people are speaking out of turn or interrupting one another, it is
appropriate for anyone to suggest a moment of \index{silence}silence to calm and
refocus energy.

\subsection{\index{taking a break}Taking a \index{break}Break}

In the heat of discussion, people are usually resistant to
interrupting the flow to \index{taking a break}take a \index{break}break, but a wise \index{facilitator}facilitator
knows, more often than not, that a five minute \index{break}break will save a
frustrating half hour or more of circular discussion and fruitless
debate.

\subsection{Call For Consensus}

The \index{facilitator}facilitator, or any member recognized to speak by the
\index{facilitator}facilitator, can call for a test for consensus. To do this, the
\index{facilitator}facilitator asks if there are any unresolved concerns which remain
unaddressed. (See section \ref{ssub:call_for_consensus}.)

\subsection{\index{summarizing}Summarizing}

The \index{facilitator}facilitator might choose to focus what has been said by
\index{summarizing}summarizing.  The summary might be made by the \index{facilitator}facilitator, the
\index{notetaker}notetaker, or anyone else appropriate. This preempts a common
problem, in which the discussion becomes circular, and one after
another, speakers repeat each other.

\subsection{\index{reformulating the proposal}Reformulating the Proposal}

After a long discussion, it sometimes happens that the proposal
becomes modified without any formal decision. The \index{facilitator}facilitator needs
to recognize this and take time to reformulate the proposal with
the new information, modifications, or deletions. Then the proposal
is presented to the group so that everyone can be clear about what
is being considered. Again, this might be done by the \index{facilitator}facilitator,
the \index{notetaker}notetaker, or anyone else.

\subsection{\index{stepping out of role}Stepping out of \index{role}Role}

If the \index{facilitator}facilitator wants to become involved in the discussion or
has strong feelings about a particular \index{agenda}agenda item, the \index{facilitator}facilitator
can step out of the \index{role}role and participate in the discussion,
allowing another member to facilitate during that time.

\subsection{\index{passing the clipboard}Passing the Clipboard}

Sometimes information needs to be collected during the \index{meeting}meeting. To
save time, circulate a clipboard to collect this information. Once
collected, it can be entered into the written record and/or
presented to the group by the \index{facilitator}facilitator.

\subsection{\index{polling}Polling (Straw Polls)}

The usefulness of \index{polling}polling within consensus is primarily
clarification of the relative importance of several issues. It is
an especially useful technique when the \index{facilitator}facilitator is confused or
uncertain about the status of a proposal and wants some clarity to
be able to suggest what might be the next process technique. Polls
are not decisions, they are non-binding referenda. All too often,
straw polls are used when the issues are completely clear and the
majority wants to intimidate the minority into submission by
showing overwhelming support rather than to discuss the issues and
resolve the concerns. Clear and simple questions are best. Polls
that involve three or more choices can be especially
manipulative. Use with discretion.

\subsection{\index{censoring}Censoring}

(This technique and the next are somewhat different from the
others. They may not be appropriate for some groups.) If someone
speaks out of turn consistently, the \index{facilitator}facilitator warns the
individual at least twice that if the interruptions do not stop,
the \index{facilitator}facilitator will declare that person \index{censoring}censored. This means the
person will not be permitted to speak for the rest of this \index{agenda}agenda
item. If the interrupting behavior has been exhibited over several
\index{agenda}agenda items, then the \index{censoring}censoring could be for a longer period of
time. This technique is meant to be used at the discretion of the
\index{facilitator}facilitator.  If the \index{facilitator}facilitator \index{censoring}censors someone and others in the
\index{meeting}meeting voice disapproval, it is better for the \index{facilitator}facilitator to step
down from the \index{role}role and let someone else facilitate, rather than get
into a discussion about the ability and judgment of the
\index{facilitator}facilitator. The rationale is the disruptive behavior makes
facilitation very difficult, is dis\index{respect}respectful and, since it is
assumed that everyone observed the behavior, the voicing of
disapproval about a \index{censoring}censoring indicates lack of confidence in the
facilitation rather than support for the disruptive behavior.

\subsection{\index{expulsion}Expulsion}

If an individual still acts very disruptively, the \index{facilitator}facilitator may
confront the behavior. Ask the person to explain the reasons for
this behavior, how it is in the best interest of the group, how it
relates to the group's purpose, and how it is in keeping with the
goals and principles. If the person is unable to answer these
questions or if the answers indicate disagreement with the common
purpose, then the \index{facilitator}facilitator can ask the individual to withdraw
from the \index{meeting}meeting.

\section{\index{group discussion techniques}Group Discussion \index{techniques}Techniques}

It is often assumed that the best form of group discussion is that
which has one person at a time speak to the \index{whole group}whole group. This is
true for some discussions. But, sometimes, other \index{techniques}techniques of
group discussion can be more productive and efficient than whole
group discussion. The following are some of the more common and
frequently used \index{techniques}techniques. These could be suggested by anyone at
the \index{meeting}meeting. Therefore, it is a good idea if everyone is familiar
with these \index{techniques}techniques. Again, be creative and adaptive.  Different
situations require different \index{techniques}techniques. Only experience
reveals how each one affects group dynamics or the best time to use
it.

\subsection{\index{identification}Identification}

It is good to address each other by name. One way to learn names
is to draw a seating plan, and as people go around and introduce
themselves, write their names on it. Later, refer to the plan and
address people by their names. In large groups, name tags can be
helpful. Also, when people speak, it is useful for them to identify
themselves so all can gradually learn each others' names.

\subsection{\index{whole group}Whole Group}

The value of \index{whole group}whole group discussion is the evolution of a group
idea. A group idea is not simply the sum of individual ideas, but
the result of the interaction of ideas during discussion. Whole
group discussion can be un\index{structure}structured and productive. It can also be
very \index{structure}structured, using various facilitation \index{techniques}techniques to focus
it. Often, \index{whole group}whole group discussion does not produce maximum
\index{participation}participation or a diversity of ideas. During \index{whole group}whole group
discussion, fewer people get to speak, and, at times, the
attitude of the group can be dominated by an idea, a mood, or a
handful of people.

\subsection{\index{small group}Small Group}

Breaking into smaller groups can be very useful. These small
groups can be dyads or triads or even larger. They can be selected
randomly or self-selected.  If used well, in a relatively short
amount of time all participants have the opportunity to share their
own point of view. Be sure to set clear time limits and select a
\index{notetaker}notetaker for each group. When the larger group reconvenes, the
\index{notetaker}notetakers relate the major points and concerns of
their group. Sometimes, \index{notetaker}notetakers can be requested to add only new
ideas or concerns and not repeat something already covered in
another report. It is also helpful for the scribe to write these
reports so all can see the cumulative result and be sure every idea
and concern gets on the list.

\subsection{\index{brainstorming}Brainstorming}
\label{sec:brainstorming}

This is a very useful technique when ideas need to be solicited
from the \index{whole group}whole group. The normal rule of waiting to speak until the
\index{facilitator}facilitator recognizes you is suspended and everyone is encouraged
to call out ideas to be written by the scribe for all to see. It is
helpful if the atmosphere created is one in which all ideas, no
matter how unusual or incomplete, are appropriate and
welcomed. This is a situation in which suggestions can be used as
catalysts, with ideas building one upon the next, generating very
creative possibilities. Avoid evaluating each other's ideas
during this time.

\subsection{\index{go-rounds}Go-rounds}

This is a simple technique that encourages \index{participation}participation. The
\index{facilitator}facilitator states a question and then goes around the room
inviting everyone to answer briefly. This is not an open
discussion. This is an opportunity to individually respond to
specific questions, not to comment on each other's responses or
make unrelated remarks.

\subsection{\index{fishbowl}Fishbowl}

The \index{fishbowl}fishbowl is a special form of \index{small group}small group discussion. Several
members representing differing points of view meet in an inner
circle to discuss the issue while everyone else forms an outer
circle and listens. At the end of a predetermined time, the whole
group reconvenes and evaluates the \index{fishbowl}fishbowl discussion. An
interesting variation: first, put all the men in the \index{fishbowl}fishbowl, then
all the women, and they discuss the same topics.

\subsection{Active Listening}
\index{active listening}
If the group is having a hard time understanding a point of view,
someone might help by active listening. Listen to the speaker, then
repeat back what was heard and ask the speaker if this accurately
reflects what was meant.

\subsection{\index{caucus}Caucusing}

A \index{caucus}caucus might be useful to help a multifaceted conflict become
clearer by unifying similar perspectives or defining specific
points of departure without the focus of the \index{whole group}whole group. It might
be that only some people attend a \index{caucus}caucus, or it might be that all
are expected to participate in a \index{caucus}caucus. The difference between
\index{caucus}caucuses and \index{small group}small groups is that \index{caucus}caucuses are composed of people
with similar viewpoints, whereas \index{small group}small group discussions are more
useful if they are made up of people with diverse viewpoints
or even a random selection of people.
