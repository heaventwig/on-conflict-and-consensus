%Spelling checked 2004-04-06
A \index{role}role is a function of process, not content. \index{role}Roles are used
during a \index{meeting}meeting according to the needs of the situation. Not all
\index{role}roles are useful at every \index{meeting}meeting, nor does each \index{role}role have to be
filled by a separate person. Formal Consensus functions more
smoothly if the person filling a \index{role}role has some experience,
therefore it is desirable to rotate \index{role}roles. Furthermore, one who has
experienced a \index{role}role is more likely to be supportive of whomever
currently has that \index{role}role. Experience in each \index{role}role also encourages
confidence and \index{participation}participation.  It is best, therefore, for the group
to encourage everyone to experience each \index{role}role.

\section{\index{agenda}\index{agenda planners}Agenda Planners}

A well planned \index{agenda}agenda is an important tool for a smooth \index{meeting}meeting,
although it does not guarantee it. Experience has shown that there
is a definite improvement in the flow and pace of a \index{meeting}meeting if
several people get together prior to the start of the \index{meeting}meeting and
propose an \index{agenda}agenda. In smaller groups, the \index{facilitator}facilitator often
proposes an \index{agenda}agenda. The \index{agenda}agenda planning committee has six tasks:

\squishitemize%\begin{itemize}

\item collect \index{agenda}agenda items
\item arrange them
\item assign presenters
\item \index{brainstorming}brainstorm discussion \index{techniques}techniques
\item assign time limits
\item write up the proposed \index{agenda}agenda
\squishend%\end{itemize}

\noindent{}There are at least four sources of \index{agenda}agenda items:

\squishitemize%\begin{itemize}

\item suggestions from members
\item reports or proposals from committees
\item business from the last \index{meeting}meeting
\item \index{standard agenda}standard \index{agenda}agenda items, including:

\squishitemizetwo%\begin{itemize}

\item \index{opening}opening
\item \index{agenda}agenda review
\item review notes
\item \index{break}break
\item announcements
\item decision review
\item \index{evaluation}evaluation
\squishend%\end{itemize}


\squishend%\end{itemize}

Once all the \index{agenda}agenda items have been collected, they are listed in
an order which seems efficient and appropriate. Planners need to be
cautious that items at the top of the \index{agenda}agenda tend to use more than
their share of time, thereby limiting the time available for the
rest. Each group has different needs. Some groups work best taking
care of business first, then addressing the difficult items. Other
groups might find it useful to take on the most difficult work
first and strictly limit the time or let it take all it needs. The
following are recommendations for keeping the focus of attention on
the \index{agenda}agenda:

\squishitemize%\begin{itemize}

\item alternate long and short, heavy and light items
\item place reports before their related proposals
%\item take care of old business before addressing new items
\item consider placing items which might generate a sense of
  accomplishment early in the \index{meeting}meeting
\item alternate presenters
\item be flexible
\squishend%\end{itemize}

Usually, each item already has a presenter. If not, assign
one. Generally, it is not wise for \index{facilitator}facilitators to present reports
or proposals. However, it is convenient for \index{facilitator}facilitators to present
some of the \index{standard agenda}standard \index{agenda}agenda items.

For complex or especially controversial items, the \index{agenda}\index{agenda planners}agenda planners
could suggest various options for \index{group discussion techniques}group discussion \index{techniques}techniques. This
may be helpful to the \index{facilitator}facilitator.

Next, assign time limits for each item. It is important to be
realistic, being careful to give each item enough time to be fully
addressed without being unfair to other items. Generally, it is not
desirable to propose an \index{agenda}agenda which exceeds the desired overall
\index{meeting}meeting time limit.

The last task is the writing of the proposed \index{agenda}agenda so all can see
it and refer to it during the \index{meeting}meeting. Each item is listed in
order, along with its presenter and time limit.

The following \index{agenda}agenda is an example of how an \index{agenda}agenda is \index{structure}structured
and what information is included in it. It shows the standard
\index{agenda}agenda items, the presenters, the time limits and the order in
which they will be considered.  It also shows one way in which
reports and proposals can be presented, but each group can
\index{structure}structure this part of the \index{meeting}meeting in whatever way suits its
needs. This model does not show the choices of \index{techniques}techniques for
group discussion which the \index{agenda}\index{agenda planners}agenda planners might have considered.
\clearpage

%\section{\index{standard agenda}Standard \index{agenda}Agenda}
\begin{figure}[hb]
\caption{\label{tab:agenda}\index{standard agenda}Standard \index{agenda}Agenda}
\begin{center}
	\begin{large}
	\begin{tabular}{lll|r}
	%
	%\newcommand{\mc}[1]{\multicolumn{2}{l}{#1}}
	%
	%\hline
	\mc{\index{agenda}\textbf{Agenda Item}} & \textbf{Presenter} & \textbf{Time} \\ \hline \hline
	\mc{\index{opening}OPENING} & \index{facilitator}Facilitator & 3 min\\ \hline
	\mc{\index{agenda}AGENDA REVIEW} & \index{facilitator}Facilitator & 5 min\\ \hline
	\mc{REVIEW NOTES} & \index{notetaker}Notetaker & 5 min\\ \hline
	\mc{REPORTS:} & & 15 min\\
	&Previous activities &&\\
	&Standing committees &&\\ \hline
	\mc{PROPOSALS:} && 15 min\\
	&\multicolumn{2}{l|}{Discussion {\footnotesize(indicate Level and technique)}}&\\ \hline
	\mc{\index{break}break} && 10 min\\ \hline
	\mc{REPORTS:} & & 10 min\\
	&Informational&&\\ \hline
	\mc{PROPOSALS} &&30 min\\
	&\multicolumn{2}{l|}{Discussion {\footnotesize(indicate Level and technique)}}&\\ \hline
	\mc{ANNOUNCEMENTS}&&5 min\\
	&Pass hat&&\\
	&Next \index{meeting}meeting&&\\ \hline
	\mc{REVIEW DECISIONS}& \index{notetaker}Notetaker & 5 min\\ \hline
	\mc{\index{evaluation}EVALUATION} && 10 min\\ \hline
	\mc{CLOSING}&\index{facilitator}Facilitator&2 min\\ \hline \hline
	\mc{TOTAL} & & 2 hours\protect\footnote{includes 5 minutes of ``facilitator flex time''}\\ \hline
	\multicolumn{4}{r}{{\scriptsize 1: Includes five minutes of ``facilitator flex time.''}}
	\end{tabular}
	\end{large}
\end{center}
\end{figure}

\clearpage

\section{\index{facilitator}Facilitator}
\label{role:facilitator}

The word facilitate means to make easy. A \index{facilitator}facilitator conducts
group business and guides the Formal Consensus process so that it
flows smoothly. Rotating facilitation from \index{meeting}meeting to \index{meeting}meeting
shares important skills among the members. If everyone has
firsthand knowledge about facilitation, it will help the flow of
all \index{meeting}meetings. Co-facilitation, or having two (or more)people
facilitate a \index{meeting}meeting, is recommended. Having a woman and a man
share the responsibilities encourages a more balanced
\index{meeting}meeting. Also, an inexperienced \index{facilitator}facilitator may apprentice with a
more experienced one. Try to use a variety of \index{techniques}techniques throughout
the \index{meeting}meeting. And remember, a little bit of humor can go a long way
in easing tension during a long, difficult \index{meeting}meeting.

Good facilitation is based upon the following principles:

\subsubsection*{\index{non-directive leadership}Non-directive Leadership}

\index{facilitator}Facilitators accept responsibility for moving through the \index{agenda}agenda
in the allotted time, guiding the process, and suggesting alternate
or additional \index{techniques}techniques. In this sense, they do lead the
group. They do not give their personal opinions nor do
they attempt to direct the content of the discussion. If they want
to participate, \emph{they must clearly relinquish the \index{role}role} and
speak as an individual. During a \index{meeting}meeting, individuals are
responsible for expressing their own concerns and
thoughts. \index{facilitator}Facilitators, on the other hand, are responsible for
addressing the needs of the group.  They need to be aware of the
group dynamics and constantly evaluate whether the discussion is
flowing well. There may be a need for a change in the discussion
technique. They need to be diligent about the fair distribution of
attention, being sure to limit those who are speaking often and
offering opportunities to those who are not speaking much or at
all. It follows that one person cannot simultaneously give
attention to the needs of the group and think about a personal
response to a given situation. Also, it is not appropriate for the
\index{facilitator}facilitator to support a particular point of view or dominate the
discussion. This does not build trust, especially in those who do
not agree with the \index{facilitator}facilitator.

\subsubsection*{\index{clarity of process}Clarity of Process}

The \index{facilitator}facilitator is responsible for leading the \index{meeting}meeting openly so
that everyone present is aware of the process and how to
participate. This means it is important to constantly review what
just happened, what is about to happen, and how it will
happen. Every time a new discussion technique is introduced,
explain how it will work and what is to be accomplished. This is
both educational and helps new members participate more fully.


\subsubsection*{\index{agenda}\index{agenda contract}Agenda Contract}

The \index{facilitator}facilitator is responsible for honoring the \index{agenda}agenda
contract. The \index{facilitator}facilitator keeps the questions and discussion
focused on the \index{agenda}agenda item. Be gentle, but firm, because fairness
dictates that each \index{agenda}agenda item gets only the time allotted. The
\index{agenda}\index{agenda contract}agenda contract is made when the \index{agenda}agenda is reviewed and
accepted. This agreement includes the items on the \index{agenda}agenda, the
order in which they are considered, and the time allotted to
each. Unless the \index{whole group}whole group agrees to change the \index{agenda}agenda, the
\index{facilitator}facilitator is obligated to keep the contract. The decision to
change the \index{agenda}agenda must be a consensus, with little or no discussion.

At the beginning of the \index{meeting}meeting, the \index{agenda}agenda is presented to the
\index{whole group}whole group and reviewed, item by item. Any member can add an item
if it has been omitted. While every \index{agenda}agenda suggestion must be
included in the \index{agenda}agenda, it does not necessarily get as much time as
the presenter wants. Time ought to be divided fairly, with
individuals recognizing the fairness of old items generally getting
more time than new items and urgent items getting more time than
items which can wait until the next \index{meeting}meeting, etc. Also, review the
suggested presenters and time limits. If anything seems
inappropriate or unreasonable, adjustments may be made. Once the
whole \index{agenda}agenda has been reviewed and consented to, the \index{agenda}agenda becomes
a contract. The \index{facilitator}facilitator is obligated to follow the order and
time limits. This encourages members to be on time to \index{meeting}meetings.


\subsubsection*{\index{good will}Good Will}

Always try to assume \index{good will}good will. Assume every statement and action
is sincerely intended to benefit the group. Assume that each
member understands the group's purpose and accepts the \index{agenda}agenda as a
contract.


Often, when we project our feelings and expectations onto others,
we influence their actions. If we treat others as though they are
trying to get attention, disrupt \index{meeting}meetings, or pick fights, they
will often fulfill our expectations. A resolution to conflict is
more likely to occur if we act as though there will be one. This is
especially true if someone is intentionally trying to cause trouble
or who is emotionally unhealthy. Do not attack the person, but
rather, assume \index{good will}good will and ask the person to explain to the group
how that person's statements or actions are in the best interest of
the group. It is also helpful to remember to separate the actor
from the action. While the behavior may be unacceptable, the person
is not \emph{bad}. Avoid accusing the person of \emph{being}
the way they behave. Remember, no one has \emph{the} answer. The
group's work is the search for the best and most creative process,
one which fosters a mutually satisfying resolution to any concern
which may arise.


\section{\index{peacekeeper}Peacekeeper}
\label{role:peacekeeper}

The \index{role}role of \index{peacekeeper}peacekeeper is most useful in large groups or when
very touchy, controversial topics are being discussed. A person who
is willing to remain somewhat aloof and is not personally invested
in the content of the discussion would be a good candidate for
\index{peacekeeper}peacekeeper. This person is selected without discussion by all
present at the beginning of the \index{meeting}meeting. If no one wants this \index{role}role,
or if no one can be selected without objection, proceed
without one, recognizing that the \index{facilitator}facilitator's job will most
likely be more difficult.


This task entails paying attention to the overall mood or tone of
the \index{meeting}meeting. When tensions increase dramatically and angers flare
out of control, the \index{peacekeeper}peacekeeper interrupts briefly to remind the
group of its common goals and \index{commitment}commitment to \index{cooperation}cooperation. The most
common way to accomplish this is a call for a few moments of
\index{silence}silence.


The \index{peacekeeper}peacekeeper is the only person with prior permission to
interrupt a speaker or speak without first being recognized by the
\index{facilitator}facilitator. Also, it is important to note that the \index{peacekeeper}peacekeeper's
comments are always directed at the \index{whole group}whole group, never at one
individual or \index{small group}small group within the larger group. Keep comments
short and to the point.


The \index{peacekeeper}peacekeeper may always, of course, point out when the group
did something well. People always like to be acknowledged for
positive behavior.


\section{Advocate}
\index{advocate}
Like the \index{peacekeeper}peacekeeper, advocates are selected without discussion at
the beginning of the \index{meeting}meeting. If, because of strong emotions,
someone is unable to be understood, the advocate is called upon to
help. The advocate would interrupt the \index{meeting}meeting, and invite the
individual to literally step outside the \index{meeting}meeting for some
one-on-one discussion. An upset person can talk to someone with
whom they feel comfortable. This often helps them make clear what
the concern is and how it relates to the best interest of the
group.  Assume the individual is acting in good faith. Assume the
concern is in the best interest of the group. While they are doing
this, everyone else might take a short \index{break}break, or continue with
other \index{agenda}agenda items. When they return, the \index{meeting}meeting (after completing
the current \index{agenda}agenda item) hears from the advocate. The intent here
is the presentation of the concern by the advocate rather than the
upset person so the other group members might hear it without the
emotional charge. This procedure is a last resort, to be used only
when emotions are out of control and the person feels unable to
successfully express an idea.


\section{\index{timekeeper}Timekeeper}
\label{role:timekeeper}

The \index{role}role of \index{timekeeper}timekeeper is very useful in almost all \index{meeting}meetings. One
is selected at the beginning of the \index{meeting}meeting to assist the
\index{facilitator}facilitator in keeping within the time limits set in the \index{agenda}agenda
contract. The skill in keeping time is the prevention of an
unnecessary time pressure which might interfere with the
process. This can be accomplished by keeping everyone aware of the
status of time remaining during the discussion. Be sure to give
ample warning toward the end of the time limit so the group can
start to bring the discussion to a close or decide to rearrange
the \index{agenda}agenda to allow more time for the current topic. There is
nothing inherently wrong with going over time as long as everyone
consents.


\section{\index{public scribe}Public Scribe}
\label{role:public_scribe}
The \index{role}role of \index{public scribe}public scribe is simply the writing, on paper or blackboard, of information for the \index{whole group}whole group to see. This person primarily assists the \index{facilitator}facilitator by taking a task which might otherwise distract the \index{facilitator}facilitator and interfere with the overall flow of the \index{meeting}meeting. This \index{role}role is particularly useful during brainstorms, report backs from \index{small group}small groups, or whenever it would help the group for all to see written information.

\section{\index{notetaker}Notetaker}
\label{role:notetaker}

The importance of a written record of the \index{meeting}meetings cannot be overstated.  The written record, sometimes called notes or minutes, can help settle disputes of memory or verify past decisions. Accessible notes allow absent members to participate in ongoing work. Useful items to include in the notes are:

\squishitemize%\begin{itemize}

\item date and attendance
\item \index{agenda}agenda
\item brief notes (highlights, statistics...)
\squishitemizetwo%\begin{itemize}
\item reports
\item discussion
\squishend%\end{itemize}


\item verbatim notes

\squishitemizetwo%\begin{itemize}

\item proposals (with revisions)
\item decisions (with concerns listed)
\item announcements
\item next \index{meeting}meeting time and place
\item \index{evaluation}evaluation comments
\squishend%\end{itemize}


\squishend%\end{itemize}

After each decision is made, it is useful to have the \index{notetaker}notetaker
read the notes aloud to ensure accuracy. At the end of the \index{meeting}meeting,
it is also helpful to have the \index{notetaker}notetaker present to the group a
review of all decisions. In larger groups, it is often useful to
have two \index{notetaker}notetakers simultaneously, because everyone, no matter how
skilled, hears information and expresses it differently. \index{notetaker}Notetakers
are responsible for making sure the notes are recorded accurately,
and are reproduced and distributed according to the desires of the
group (e.g., mailed to everyone, handed out at the next \index{meeting}meeting,
filed, etc.).


\section{\index{doorkeeper}Doorkeeper}
\label{role:greeter}

\index{doorkeeper}Doorkeepers are selected in advance of the \index{meeting}meeting and need to
arrive early enough to familiarize themselves with the physical
layout of the space and to receive any last minute instructions
from the \index{facilitator}facilitator. They need to be prepared to miss the first
half hour of the \index{meeting}meeting. Prior to the start of the \index{meeting}meeting, the
\index{doorkeeper}doorkeeper welcomes people, distributes any literature connected to
the business of the \index{meeting}meeting, and informs them of any pertinent
information (the \index{meeting}meeting will start fifteen minutes late, the
bathrooms are not wheelchair accessible, etc.).


A \index{doorkeeper}doorkeeper is useful, especially if people tend to be late. When
the \index{meeting}meeting begins, they continue to be available for
late comers. They might briefly explain what has happened so far and
where the \index{meeting}meeting is currently on the \index{agenda}agenda. The \index{doorkeeper}doorkeeper might
suggest to the late comers that they refrain from participating in
the current \index{agenda}agenda item and wait until the next item before
participating. This avoids wasting time, repeating discussion, or
addressing already resolved concerns. Of course, this is not a
rigid rule. Use discretion and be \index{respect}respectful of the group's time.


Experience has shown this \index{role}role to be far more useful than it might
at first appear, so experiment with it and discover if \index{meeting}meetings can
become more pleasant and productive because of the friendship and
care which is expressed through the simple act of greeting people
as they arrive at the \index{meeting}meeting.

