%

The fundamental difference between consensus and voting is one of cooperation and competition. This affects the structure of the process as well as the attitude of the participants. Consensus fosters an environment in which everyone is respected and all contributions are valued. Creative resolution of all concerns contributes to the overall quality of the decision. Voting encourages competition, often without regard to others concerns, since its goal is the winning of the most votes. Using majority rule risks alienation and apathy within the group.

Consensus requires a different kind of attitude toward conflict and its resolution. Conflict is considered necessary, welcomed, and desirable, not something to be avoided, repressed, or feared. Its resolution is achieved through creativity and cooperation. The groups strives to create an environment in which disagreement can be expressed without fear and heard as a concern which, when resolved, will make the decision stronger.

This attitude opposes our socialized attitude towards conflict. It is challenging to invite conflict into our discussions. Creation of a cooperative, supportive environment will require tolerance and a willingness to experiment. In the early stages, this might prove to be difficult; pent-up frustrations and unexpressed angers based upon concerns that were never before allowed to be expressed will surface. However, if groups stay with this process, they will be rewarded with improved group dynamics, more creative resolutions, and greater trust and respect.

Since the skills and techniques necessary for consensus process are not readily taught in our society, it is unreasonable to expect any group to be able to use consensus without first taking the time to learn it. It is also important to recognize that not only do new skills, techniques, and language need to be learned but, in addition, the old habits of competitiveness, defensiveness, and possessiveness inherent in parliamentary procedure need to be ``unlearned.'' Specific attention must be given to the fact that almost all of us have been taught to behave in these ways and this undermines our ability to use consensus. For this reason, the consensus process known as Formal Consensus was created.

Formal Consensus employs a clearly defined yet flexible structure, a rigid agenda contract, and strong facilitation. Often, groups use consensus without ever agreeing upon a particular way of using it. Therefore, from week to week and month to month, the process changes without conscious effort. This can lead to frustration and manipulation. If the process is not clearly defined, access to decision making can be very difficult for members (especially new members). In any given meeting, if the agenda is not honored as a rigid contract, earlier items will get an unfair amount of the meeting time and later items will be shortened, possibly missed altogether. The same thing can happen within one agenda item. During discussion, one idea or one person can dominate the time, not allowing other ideas or all members an opportunity to be heard. A strong facilitator can recognize this and apply facilitation techniques that more fairly distribute the attention of the group. It should be noted that, while the facilitator may be powerful with regard to process, the facilitator should not be involved with or comment on the content of any particular agenda item. If this happens, the facilitator should ``step out of the role.''

Formal Consensus is efficient and effective. It provides a clearly defined structure so that even the most complicated decision can be made calmly and timely. But to accomplish this, it also requires training and discipline. Formal Consensus is nonviolent, democratic, based upon the group's principles, better in larger groups, better when everyone participates, not inherently time-consuming, and cannot be secretly disrupted (that is, the structure of this process reveals hidden agendas).

\vspace{1in}

%\noindent
\begin{center}{\scriptsize[This article first appeared in Co-op News Network (May/June 1993).]}\end{center}
%\vfill