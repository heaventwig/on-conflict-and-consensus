%Spelling checked 2004-04-01
If you were asked to pick one thing that might bring about major
social, political, and economic change in this country, what would
you pick?  Most people would pick their favorite issue; be it
civil rights, demilitarization, environmental sustainability, or
whatever.  Some people would choose a system of values to replace
the capitalism system such as socialism or the Ten Key Values of
the Greens.  But few people would even think of changing group
dynamics (the way people treat each other when interacting with
one another in a group); or specifically, the process they use
when making decisions.

Process is the key to revolutionary change.  This is not a new
message.  Visionaries have long pointed to this but it is a hard
lesson to learn.  As recently as the 70s, feminists clearly
defined the lack of an alternative process for \index{decisionmaking}decisionmaking and
group interaction as the single most important obstacle in the way
of real change, both within progressive organizations and for
society at large.  Despite progress on many issues of concern to
progressive-minded people, very little has changed in the way
people treat each other, either locally or globally, and almost
nothing has changed about who makes the decisions.  The values of
competition, which allow us to accept the idea that somebody has
to lose; the \index{structure}structure of hierarchy, which, by definition, creates
\index{power}power elites; and the \index{techniques}techniques of domination and control, which
dehumanizes and alienates all parties affected by their use, are
the standards of group interaction with which we were all
conditioned.  There are but a few models in our society which
offer an alternative.

All groups, no matter what their mission or political philosophy,
use some form of process to accomplish their work.  Almost all
groups, no matter where they fall on the social, political, and
economic spectrum of society, have a hierarchical \index{structure}structure,
accept competition as ``natural'', acceptable, and even
desirable, and put a good deal of effort into maintaining control
of their members.  It is telling that in our society, there are
opposing groups, with very different perspectives and values,
which have identical \index{structure}structures and \index{techniques}techniques for interaction and
\index{decisionmaking}decisionmaking.  If you played a theater game in which both groups
wore the same costumes and masks and spoke in gibberish rather
than words, a spectator would not be able to tell them apart.

So what would an alternative revolutionary \index{decisionmaking}decisionmaking process
look like, you ask?  To begin with, a fundamental shift from
competition to \index{cooperation}cooperation.  This does not mean to do away with
competition.  Ask any team coach what the key to victory is and
you will be told ``\index{cooperation}cooperation within the team''.  The
fundamental shift is the use of competition not to win, which is
just a polite way of saying to dominate, to beat, to destroy, to
kill the opposition; but rather, to use competition to do or be
the best.  In addition, the cooperative spirit recognizes that it
is not necessary to attack another's efforts in order to do your
best; in fact, the opposite is true.  In most situations, helping
others do their best actually increases your ability to do better.
And in group interactions, the cooperative spirit actually allows
the group's best to be better than the sum of its parts.

\index{cooperation}Cooperation is more than ``live and let live''.  It is
making an effort to understand another's point of view.  It is
incorporating another's perspective with your own so that a new
perspective emerges.  It is suspending disbelief, even if only
temporarily, so you can see the gem of truth in ideas other than
your own.  It is a process of creativity, synthesis, and
open-mindedness which leads to trust-building, better
communication and understanding, and ultimately, a stronger,
healthier, more successful group.

The next step is the development of an organization which is
non-hierarchical or egalitarian.  A corresponding \index{structure}structure would
include: participatory democracy, routine universal skill-building
and information sharing, rotation of leadership \index{role}roles, frequent
\index{evaluation}evaluations, and, perhaps most importantly, \index{equal access to power}equal access to \index{power}power.
Hierarchical \index{structure}structures are not, in and of themselves, the
problem.  But their use concentrates \index{power}power at the top and,
invariably, the top becomes less and less accessible to the people
at the bottom, who are usually most affected by the decisions made
by those at the top.  Within groups (and within society itself),
there becomes a \index{power}power elite.  In an egalitarian \index{structure}structure,
everyone has access to \index{power}power and every position of \index{power}power is
accountable to everyone.  This does not mean that there are no
leaders.  But the leaders actively share skills and information.
They recognize that leadership is a \index{role}role empowered by the entire
group, not a personal characteristic.  A group in which most or
all of the members can fill any of the leadership \index{role}roles cannot
easily be dominated, internally or externally.

The last and most visible step toward revolutionary change in
group process is the manner in which members of the group interact
with each other.  Dominating attitudes and controlling behavior
would not be tolerated.  People would show \index{respect}respect and expect to
be shown \index{respect}respect.  Everyone would be doing their personal best to
help the group reach decisions which are in the best interest of
the group.  There would be no posturing and taking sides.
Conflicts would be seen as an opportunity for growth, expanding
people's thinking, sharing new information, and developing new
solutions which include everyone's perspectives.  The group would
create an environment where everyone was encouraged to
participate, conflict was freely expressed, and resolutions were
in the best interest of everyone involved.  Indubitably, this
would be revolutionary.
