%Spelling checked 2004-04-01
Conflict is usually viewed as an impediment to reaching agreements and disruptive to peaceful relationships. However, it is the underlying thesis of Formal Consensus that nonviolent conflict is necessary and desirable. It provides the \emph{motivations} for improvement. The challenge is the creation of an understanding in all who participate that conflict, or differing opinions about proposals, is to be expected and acceptable. Do not avoid or repress conflict. Create an environment in which disagreement can be expressed without fear. Objections and criticisms can be heard not as attacks, not as attempts to defeat a proposal, but as a concern which, when resolved, will make the proposal stronger.

This understanding of conflict may not be easily accepted by the members of a group. Our training by society undermines this concept. Therefore, it will not be easy to create the kind of environment where differences can be expressed without fear or resentment. But it can be done. It will require tolerance and a willingness to experiment. Additionally, the values and principles which form the basis of \index{commitment}commitment to work together to resolve conflict need to be clearly defined, and accepted by all involved.

If a group desires to adopt Formal Consensus as its \index{decisionmaking}decisionmaking process, the first step is the creation of a \emph{Statement of Purpose} or \emph{Constitution}. This document would describe not only the common purpose, but would also include the definition of the group's principles and values. If the group discusses and writes down its foundation of principles at the start, it is much easier to determine group versus individual concerns later on.

The following are principles which form the foundation of Formal Consensus. A \index{commitment}commitment to these principles and/or a willingness to develop them is necessary. In addition to the ones listed herein, the group might add principles and values which are specific to its purpose.

\section[Foundations of Consensus]{Foundation Upon Which\\ Consensus Is Built}

For consensus to work well, the process must be conducted in an environment which promotes trust, \index{respect}respect, and skill sharing. The following are principles which, when valued and \index{respect}respected, encourage and build consensus.

\subsection*{Trust}

Foremost is the need for trust. Without some amount of trust, there will be no \index{cooperation}cooperation or nonviolent resolution to conflict. For trust to flourish, it is desirable for individuals to be willing to examine their attitudes and be open to new ideas. Acknowledgment and appreciation of personal and cultural differences promote trust. Neither approval nor friendship are necessary for a good working relationship. By developing trust, the process of consensus encourages the intellectual and emotional development of the individuals within a group.

\subsection*{\index{respect}Respect}

It is everyone's responsibility to show \index{respect}respect to one another. People feel \index{respect}respected when everyone listens, when they are not interrupted, when their ideas are taken seriously. \index{respect}Respect for emotional as well as logical concerns promotes the kind of environment necessary for developing consensus. To promote \index{respect}respect, it is important to distinguish between an action which causes a problem and the person who did the action, between the deed and the doer. We must criticize the act, not the person. Even if you think the person \emph{is} the problem, responding that way never resolves anything.% (See page \pageref{sub:formal_consensus_is_the_least_violent_decisionmakingdecisionmaking_process_}.)

\subsection*{\index{unity of purpose}Unity of Purpose}

\index{unity of purpose}Unity of purpose is a basic understanding about the goals and purpose of the group. Of course, there will be varying opinions on the best way to accomplish these goals. However, there must be a unifying base, a common starting point, which is recognized and accepted by all.

\subsection*{Nonviolence}

Nonviolent decisionmakers use their \index{power}power to achieve goals while \index{respect}respecting differences and cooperating with others. In this environment, it is considered violent to use \index{power}power to dominate or control the group process. It is understood that the \index{power}power of revealing your truth is the maximum force allowed to persuade others to your point of view.

\subsection*{Self Empowerment}

It is easy for people to unquestioningly rely on authorities and experts to do their thinking and \index{decisionmaking}decisionmaking for them. If members of a group delegate their authority, intentionally or not, they fail to accept responsibility for the group's decisions. Consensus promotes and depends upon self empowerment. Anyone can express concerns. Everyone seeks creative solutions and is responsible for every decision. When all are encouraged to participate, the democratic nature of the process increases.

\subsection*{\index{cooperation}Cooperation}

Unfortunately, Western society is saturated in competition. When winning arguments becomes more important than achieving the group's goals, \index{cooperation}cooperation is difficult, if not impossible. Adversarial attitudes toward proposals or people focus attention on weakness rather than strength. An attitude of helpfulness and support builds \index{cooperation}cooperation. \index{cooperation}Cooperation is a shared responsibility in finding solutions to all concerns. Ideas offered in the spirit of \index{cooperation}cooperation help resolve conflict. The best decisions arise through an open and creative interplay of ideas.

\subsection*{Conflict Resolution}

The free flow of ideas, even among friends, inevitably leads to conflict. In this context, conflict is simply the expression of disagreement. Disagreement itself is neither good nor bad. Diverse viewpoints bring into focus and explore the strengths and weaknesses of attitudes, assumptions, and plans. Without conflict, one is less likely to think about and evaluate one's views and prejudices. There is no \emph{right} decision, only the best one for the \index{whole group}whole group. The task is to work together to discover which choice is most acceptable to all members.

Avoid blaming anyone for conflict. Blame is inherently violent. It attacks dignity and \index{empowerment}empowerment. It encourages people to feel guilty, defensive, and alienated. The group will lose its ability to resolve conflict. People will hide their true feelings to avoid being blamed for the conflict.

Avoidance of conflicting ideas impedes resolution for failure to explore and develop the feelings that gave rise to the conflict. The presence of conflict can create an occasion for growth. Learn to use it as a catalyst for discovering creative resolutions and for developing a better understanding of each other. With \index{patience}patience, anyone can learn to resolve conflict creatively, without defensiveness or guilt. Groups can learn to nurture and support their members in this effort by allowing creativity and experimentation. This process necessitates that the group continually evaluate and improve these skills.

\subsection*{\index{commitment}Commitment to the Group}

In joining a group, one accepts a personal responsibility to behave with \index{respect}respect, \index{good will}good will, and honesty. Each one is expected to recognize that the group's needs have a certain priority over the desires of the individual. Many people participate in group work in a very egocentric way. It is important to accept the shared responsibility for helping to find solutions to other's concerns.

\subsection*{Active \index{participation}Participation}
\index{active participation}
We all have an inalienable right to express our own best thoughts. We decide for ourselves what is right and wrong. Since consensus is a process of synthesis, not competition, all sincere comments are important and valuable. If ideas are put forth as the speaker's property and individuals are strongly attached to their opinions, consensus will be extremely difficult. Stubbornness, closed-mindedness, and possessiveness lead to defensive and argumentative behavior that disrupts the process. For active \index{participation}participation to occur, it is necessary to promote trust by creating an atmosphere in which every contribution is considered valuable. With encouragement, each person can develop knowledge and experience, a sense of responsibility and competency, and the ability to participate.

\subsection*{\index{equal access to power}Equal Access to \index{power}Power}

Because of personal differences (experience, assertiveness, social conditioning, access to information, etc.) and political disparities, some people inevitably have more effective \index{power}power than others. To balance this inequity, everyone needs to consciously attempt to creatively share \index{power}power, skills, and information. Avoid hierarchical \index{structure}structures that allow some individuals to assume undemocratic \index{power}power over others. Egalitarian and accountable \index{structure}structures promote universal access to \index{power}power.

\subsection*{\index{patience}Patience}

Consensus cannot be rushed. Often, it functions smoothly, producing effective, stable results. Sometimes, when difficult situations arise, consensus requires more time to allow for the creative interplay of ideas. During these times, \index{patience}patience is more advantageous than tense, urgent, or aggressive behavior. Consensus is possible as long as each individual acts patiently and \index{respect}respectfully.

\section{Impediments To Consensus}
\subsection*{Lack of Training}

It is necessary to train people in the theory and practice of consensus. Until consensus is a common form of \index{decisionmaking}decisionmaking in our society, new members will need some way of learning about the process. It is important to offer regular opportunities for training. If learning about Formal Consensus is not made easily accessible, it will limit full \index{participation}participation and create inequities which undermine this process. Also, training provides opportunities for people to improve their skills, particularly facilitation skills, in a setting where experimentation and \index{role}role-plays can occur.

\subsection*{External Hierarchical \index{structure}Structures}

It can be difficult for a group to reach consensus internally when it is part of a larger group which does not recognize or participate in the consensus process. It can be extremely frustrating if those external to the group can disrupt the \index{decisionmaking}decisionmaking by interfering with the process by pulling rank. Therefore, it is desirable for individuals and groups to recognize that they can be autonomous in relation to external \index{power}power if they are willing to take responsibility for their actions.

\subsection*{\index{social prejudice}Social Prejudice}

Everyone has been exposed to biases, assumptions, and prejudices which interfere with the spirit of \index{cooperation}cooperation and equal \index{participation}participation. All people are influenced by these attitudes, even though they may deplore them. People are not generally encouraged to confront these prejudices in themselves or others. Members of a group often reflect social biases without realizing or attempting to confront and change them. If the group views a prejudicial attitude as just one individual's problem, then the group will not address the underlying social attitudes which create such problems. It is appropriate to expose, confront, acknowledge, and attempt to resolve socially prejudicial attitudes, but only in the spirit of mutual \index{respect}respect and trust. Members are responsible for acknowledging when their attitudes are influenced by disruptive social training and for changing them. When a supportive atmosphere for recognizing and changing undesirable attitudes exists, the group as a whole benefits.

